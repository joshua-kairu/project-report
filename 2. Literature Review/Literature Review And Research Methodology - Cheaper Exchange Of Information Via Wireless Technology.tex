
\documentclass[12pt,svgnames,smaller]{article} % use larger type; default would be 10pt

\usepackage[table]{xcolor}
\usepackage[utf8]{inputenc} % set input encoding (not needed with XeLaTeX)
\usepackage[pdftex]{graphicx}

\usepackage{geometry} % to change the page dimensions
\geometry{a4paper} % or letterpaper (US) or a5paper or....

\usepackage{graphicx} % support the \includegraphics command and options

\usepackage[parfill]{parskip} % Activate to begin paragraphs with an empty line rather than an indent

\usepackage{verbatim} % adds environment for commenting out blocks of text & for better verbatim
\usepackage{mathrsfs}

\usepackage{amssymb}
\usepackage{amsthm}
\newtheorem{myalg}{Algorithm}
\newtheorem{mydef}{Definition}
\newtheorem{myprop}{Proposition}
\newtheorem{myproof}{Proof}
\newtheorem{myexample}{Example}

\usepackage{amsmath}
\newcommand{\E}[1]{\textnormal{\textsf{E}}\!\left[#1\right]} % expectation

\usepackage{tabularx}

%\usepackage{tabulary}

\renewcommand{\labelenumi}{\alph{enumi})} % change the listing of the first level of lists to a) b) c)

\renewcommand{\labelenumii}{\roman{enumii}.} % change the listing of the second level of lists to i. ii. iii.

\renewcommand{\labelenumiii}{(\alph{enumiii})} % change the listing of the third level of lists to (a) (b) (c)

\newenvironment{xitemize}
{\begin{minipage}[t]{0.5\textwidth}\begin{itemize}}
		{\end{itemize}\end{minipage}}

\begin{document}
	\begin{titlepage}
		\begin{center}
			
			% Upper part of the page
			% ======================
			
			\textsc{\large Jomo Kenyatta University of Agriculture and Technology (JKUAT)}\\[0.4cm]
			%\textsc{\Large College of Pure and Applied Science (COPES)}\\[1.0cm]
			%\textsc{\Large School of Computing and Information Technology (SCIT)}\\[0.5cm]
			\textsc{\large Department of Computing}\\[0.4cm]
			\textsc{\large ICS2406: Computer Systems Project  }\\[0.3cm]
			\textsc{\large Literature Review and Research Methodology }\\[0.3cm]
			\textsc{\large REF:JKU/2/83/022 }\\[0.2 cm]
			
			% Title
			% =====
			
			\begin{figure}
				\centering
				\includegraphics[width=0.7\linewidth]{"/run/media/joshua/DATA/ICS_2406_Pictures/Base/JKUAT_Logo"}
				\label{fig:JKUAT_logo}
			\end{figure}
			\textbf{\texttt{ Project Title: Cheaper Exchange of Information Via Wireless Technology }}{\vspace*{20 mm}}
		\end{center}
		
		% Author and supervisors
		% ======================
		
		\begin{figure}
			\centering
			\includegraphics[width=0.7\linewidth]{"/run/media/joshua/DATA/ICS_2406_Pictures/Base/JKUAT_Logo"}
			\label{fig:JKUAT_logo}
		\end{figure}
		
		
		\begin{minipage}{1.0\textwidth}	
			\begin{flushleft} \large
				\emph{Author:}\\
				Name: Kairu \textsc{Joshua Wambugu} Reg. No: CS281-0720/2011 \\
				Submission Date: \hrulefill  Sign: \hrulefill \\
				Course: Bsc. Computer Science
			\end{flushleft}
		\end{minipage}
		
		\begin{minipage}{1.0\textwidth}
			\begin{flushleft} \large
				\emph{Supervisor 1:} \\
				Name: Professor \textsc{Waweru Mwangi} Sign: \hrulefill Date: \hrulefill
			\end{flushleft}
		\end{minipage}
		
		\begin{minipage}{1.0\textwidth}
			\begin{flushleft} \large
				\emph{Supervisor 2:} \\
				Name: Doctor \textsc{Petronilla Muthoni} Sign: \hrulefill Date: \hrulefill
			\end{flushleft}
		\end{minipage}
		
		\begin{minipage}{1.0\textwidth}
			\begin{flushleft} \large
				\emph{Supervisor 3:} \\
				Name: \hrulefill Sign: \hrulefill Date: \hrulefill
			\end{flushleft}
		\end{minipage}
		
		\vfill
		
		% Bottom of the page
		% ==================
		
		\begin{center}
			{\large Period: June 2015}
		\end{center}
		%{\large \today}
		%\end{center}
	\end{titlepage}
	
	%\maketitle
	
	\clearpage
	\tableofcontents
	\clearpage

	% begin section Lists of Figures and Tables
	% ============================================
	
	\section*{\textbf{ Lists of Figures and Tables}}
	
	\addcontentsline{toc}{section}{\protect\numberline{} \textbf{ Lists of Figures and Tables}} % ensure the lists of figures and tables do not affect the toc
	
	\listoffigures
	
	\listoftables
	
	\clearpage

	% end section Lists of Figures and Tables
	% =======================================
		
	% begin Definition of Abbreviations
	% =================================
	
	\section*{ \textbf{ Definition of Abbreviations} }
	
	\addcontentsline{toc}{section}{\protect\numberline{} \textbf{ Definition of Abbreviations}} % ensure the definitions of abbreviations do not affect the toc

	% begin itemize abbreviations
	\begin{itemize}
		\item 3G – 3rd Generation.
		\item 3GPP – 3rd Generation Partnership Project.
		\item AAC – Advanced Audio Coding.
		\item ACELP – Algebraic Code Excited Linear Prediction. 
		\item AES-CCMP – Advanced Encryption Standard Counter Block Chaining Message Authentication Protocol.
		\item AMR – Adaptive Multi-Rate.
		\item AMR-NB – Adaptive Multi-Rate Narrowband. 
		\item AMR-WB – Adaptive Multi-Rate Wideband.
		\item AP – Access Point.
		\item AUC – Authentication Center.
		\item BSC – Base Station Controller. 
		\item BSS – Base Station System.
		\item BTS – Base Transceiver Station.
		\item CDMA – Code Division Multiple Access. 
		\item CELP – Code Excited Linear Prediction.
		\item CNG – Comfort Noise Generation.
		\item DHCP – Dynamic Host Configuration Protocol.
		\item DTX – Discontinuous Transmission.
		\item ECC – Electronic Communications Committee.
		\item EDGE – Enhanced GSM Data Environment.
		\item EIR – Equipment Identity Register.
		\item ETSI – European Telecommunications Standards Institute.
		\item GSM – Global System for Mobile communications.
		\item HDTV – High Definition Television.
		\item HLR – Home Location Register.
		\item IDE – Integrated Development Environment.
		\item IEC – International Engineering Consortium.
		\item IMDCT – Inverse Modified Discrete Cosine Transform.
		\item IP – Internet Protocol.
		\item Kbps – Kilobits per second.
		\item MDCT – Modified Discrete Cosine Transform.
		\item ME – Mobile Equipment.
		\item MP3 – Motion Picture Experts Group-1/2 Layer-3.
		\item MPEG – Motion Picture Experts Group. 
		\item MSC – Mobile services Switching Center.
		\item NFC – Near Field Communication.
		\item OoBTC – Out-of-Band Transcoder Control.
		\item OSS – Operation and Support System.
		\item P2P – Peer-To-Peer.
		\item P2P GO – P2P Group Owner.
		\item PCM – Pulse Code Modulation.
		\item PCMCIA – Personal Computer Memory Card International Association.
		\item PIN – Personal Identification Number
		\item PSK – Pre-Shared Key.
		\item QoS – Quality of Service.
		\item SIM – Subscriber Identity Module.
		\item SS – Switching System.
		\item TCP – Transmission Control Protocol.
		\item TCP/IP – Transmission Control Protocol/Internet Protocol. 
		\item TDMA – Time Division Multiple Access.
		\item TFO – Tandem-Free Operation.
		\item TNS – Temporal Noise Shaping. 
		\item TrFO – Transcoder-Free Operation. 
		\item USB – Universal Serial Bus.
		\item UTRAN – Universal Terrestrial Radio Access Network.
		\item VAD – Voice Activity Detector.
		\item VLR – Visitor Location Register.
		\item WCDMA – Wide Code Division Multiple Access.
		\item Wi-Fi – Wireless Fidelity.
		\item WLAN – Wireless Local Area Network.
		\item WPA2 – Wi-Fi Protected Access II.
		\item WPS – Wi-Fi Protected Security. 
	\end{itemize}
	% end itemize abbreviations
	
	% end Definition of Abbreviations
	% ===============================
	
	\clearpage
		
	% begin Background/Introduction
	% =============================
	
	\section{ \textbf{ Background/Introduction } }
	
	As established in the Project Proposal, communication has developed a lot over the decades. Currently, cell phones work by creating a connection between two individual mobile phones. That connection is done by allowing the two devices to use their network service provider infrastructure. The use of that infrastructure is paid for by the users of the mobile phones. 
	The Project Proposal also established that smartphones have had a major impact on communication. They have become handheld computers. Computers can communicate with each other over both wired and wireless networks without use of third party infrastructure. Could the same be done between smartphones? This could be tested by seeing if two smartphones can be connected via wireless without a third party. Once this is established, one can try to send and receive data via this connection. Doing so will ensure that two smartphones will communicate with each other without incurring the costs of involving a third party --– reducing the costs of information sharing among smartphones. The information the project will try to send and receive is audio information –-- thus simulating a phone call. 
	
	To achieve these goals, we need knowledge concerning wireless technologies, peer-to-peer implementations, and audio encoding and file formats. 
	
	This literature review tries to analyse the aforementioned items.
	
	% end Background/Introduction
	% ===========================
	
	
	% begin Literature Review
	% =======================
	
	\section{ \textbf{ Literature Review } } 
	
	This literature review will focus on the following items.
	
	\begin{enumerate}
		\item Three wireless technologies --- NFC, Bluetooth, and WiFi;
		\item Two peer-to-peer technologies --- SMPP and WiFi P2P;
		\item Two audio encoding techniques --- AMR-NB and AMR-WB; and
		\item Two audio file formats --- MP3 and AAC.
	\end{enumerate}
	
	\subsection{\textbf{Wireless Technologies}}
	
	A number of wireless technologies were mentioned in the project proposal. These were:
	
	\begin{enumerate}
		\item Near Field Communication(NFC) technology;
		\item Wireless Fidelity (Wi-Fi) technology; and
		\item Bluetooth technology. 
	\end{enumerate}
	
	These technologies will be considered to some detail below.
	
	\begin{enumerate}
		\item \textbf{ Near Field Communication(NFC) technology }
		
		According to a programmer’s guide to Android (Deitel et al. 2012, p. 11), Near Field Communication, or NFC, is a short-range radio frequency (RF) wireless connectivity standard that enables communication between two devices. It can also be used between a device and a tag --– which stores data that can be read by NFC-enabled devices. NFC operates within a range of a few centimetres. NFC-enabled gadgets can operate in three modes:
		
		\begin{enumerate}
			\item \textbf{Reader/writer} --– such as when a device reads data from a tag (Deitel et al. 2012);
			\item \textbf{Peer to peer} --- where devices exchange information without involving a third party server (Deitel et al. 2012); and
			\item \textbf{Card emulation} --– where devices act like smart cards, accomplishing various smart card operations (Deitel et al. 2012).
		\end{enumerate}
		
		Currently, Android devices support reader/writer and peer-to-peer NFC modes.
		
		According to the ISO NFC standard governing NFC protocols (ISO/IEC-18092 2013), NFC devices can have one of the following roles in an NFC network:
		
		\begin{enumerate}
			\item \textbf{Initiator} --– that is, the generator of the RF field in which NFC signals will be passed between the communicating devices. The Initiator is also the starter of NFC communication.
			\item \textbf{Target} --- which responds to Initiator commands either using RF generated by the Initiator through a method is called the load modulation scheme; or using modulation of an RF field generated by the Target itself. 
		\end{enumerate}
		
		The same ISO standard (ISO/IEC-18092 2013) defines two modes the Initiator and Target devices can communicate with:
		
		\begin{enumerate}
			\item \textbf{Active communication mode} --– in which both the Initiator and the Target devices use their own RF field to enable communication; and
			\item \textbf{Passive communication mode} –-- where the Initiator generates the RF field and the Target responds to an Initiator command in a load modulation scheme.
		\end{enumerate}
		
		NFC Targets and Initiators usually have an implementation of both the Active and Passive communication modes. 
		
		According to the aforementioned standard (ISO/IEC-18092 2013),transactions between NFC devices start with device initialization. Initiators select one of three bit rates ($\tfrac{frequency of RF field}{128}$,$\tfrac{frequency of RF field}{64}$, or $\tfrac{frequency of RF field}{32}$ bits) to start the transaction. Initiators may change the bit rate in the middle of the transaction using certain commands. However, communication modes --- Active or Passive --- cannot be changed during one transaction. 
		
		\begin{figure}
			\centering
			\includegraphics[width=0.7\linewidth]{"/run/media/joshua/DATA/ICS_2406_Pictures/01LiteratureReviewAndResearchMethodology/LR_F1_NFC_Communication_Procedures"}
			\caption{NFC Communication Procedures}
			\label{fig:LiteratureReview-Figure1}
		\end{figure}
		
		Figure \ref{fig:LiteratureReview-Figure1} gives a simple illustration how NFC communication usually happens. The figure is a simple one and leaves out some details of what is involved during each step of NFC communication since a lot of smaller processes come into play during NFC setup and teardown. The following list touches just a few of these processes:
		
		\begin{itemize}
			\item \textbf{RF collision avoidance} – which checks to ensure an Initiator communicates with only one Target at a time.
			\item \textbf{The Single Device Detection (SDD) algorithm} – which is used by the Initiator to detect one out of several Targets in the Initiator’s RF field.
			\item \textbf{Active or Passive mode activation} – which is done using the Attribute Request and Attribute Response messages. (Both messages are called ATRs) Activation also involves the NFCID3 – a random ID used to finish the transport protocol activation process.
			\item \textbf{Parameter selection} – which is done using the Parameter Selection Request (PSL\_REQ) and Parameter Selection Response (PSL\_RES) commands.
			\item \textbf{Data Exchange Protocol selection} – which is done using the Data Exchange Protocol Request (DEP\_REQ) and the Data Exchange Protocol Response (DEP\_RES) commands.
			\item \textbf{Deselection  and release} – which are done during device deactivation. 
		\end{itemize}
		
		According to an article in the April 2012 issue of the International Journal of Advanced Research in Computer Science and Software Engineering (Preethi, Sinha, \& Varma 2012), NFC has a range of up to 10 centimetres. (This translates to roughly 4 inches)
		
		The NFC Forum, which champions NFC technology, was founded in 2004 but had to wait until 2006 before NFC tags came on the scene.
		
		NFC-enabled devices include, but are not limited to, credit cards, smart posters, smart phones, and even on some computers.
		
		The following are two advantages of using NFC;
		
		\begin{itemize}
			\item NFC provides security since its range is quite small. Piggybacking – which, according to a Computer Science journal article (Arul Oli, 2013), is the situation where unauthorized devices can access a wireless network by virtue of being within the operating range of that network – is almost impossible with NFC. This is because NFC operates within a very small range. Intruders would have to be very close to the victim devices to access them via NFC.
			\item NFC helps make device use intuitive. In English, “communicate” can mean “get in touch.” NFC helps two devices communicate by getting in touch. The concept is thus instinctive and therefore easy to adapt to daily life.
		\end{itemize}
		
		The following are two disadvantages of NFC;
		
		\begin{itemize}
			\item The NFC technology is relatively new. It is not common. Anecdotally, relatively few people have NFC enabled smart phones in Kenya.
			\item NFC can only transfer small quantities of data. It does not work well with transfer of data in the millions of bytes. This is because NFC has a relatively small maximum transfer rate of 424 kilobits per second. (Preethi, Sinha, \& Varma, 2012)	
		\end{itemize}
		
		\item \textbf{ Bluetooth technology }
		
		A study of Bluetooth (Singh, Sharma, \& Agrawal, 2011) defined Bluetooth as a wireless communication protocol aimed at low-powered, short range applications. It was initially developed by Ericsson but is now governed by the Bluetooth Special Interest Group (SIG). Initially, it was proposed as a technology to replace cables among computer components --- think of a computer’s monitor, motherboard, mouse, and keyboard working seamlessly without having to be connected via physical cabling. Bluetooth has grown past that goal --- in part due to its low power consumption and potential low cost.
		
		The Bluetooth we know today started in Scandinavia around 1996 when a certain Jim Kardach developed a system to allow mobile phones to communicate with computers. The name Bluetooth is based on the tenth-century Scandinavian king known in English as Harald Bluetooth. He united the whole of Denmark, achieving with the Danes what Kardach and his colleagues intended to with computers and cell phones.  
		
		According to a survey on Bluetooth security, (Ibn Minar \& Tarique, 2012), Bluetooth was officially approved in the summer of 1999. Since then, the Bluetooth SIG has grown to have over 14,00 members, including some leading companies in telecommunications, computing, automotive, music, industrial automation, and network industries.
		
		The same survey noted that Bluetooth is a combination of both hardware and software. On the one hand, the hardware is placed on a radio chip. On the other hand, the main control and security protocols have been implemented as software.
		
		\begin{figure}
			\centering
			\includegraphics[width=0.7\linewidth]{"/run/media/joshua/DATA/ICS_2406_Pictures/01LiteratureReviewAndResearchMethodology/LR_F2_Bluetooth_Protocol_Stack"}
			\caption{Bluetooth Protocol Stack}
			\label{fig:LiteratureReview-Figure2}
		\end{figure}
		
		As per the previously mentioned Bluetooth security article, Bluetooth support involves both hardware and software. The hardware rides on a radio chip while software is used to implement control and security protocols. Using both hardware and software makes Bluetooth quite flexible. Over the next few paragraphs we will consider the hardware and software parts of Bluetooth.
		
		Figure \ref{fig:LiteratureReview-Figure2} – adapted from Ibn Minar and Tarique (2012) –  shows the Bluetooth protocol stack. According to a study by those two (Ibn Minar \& Tarique 2012), a protocol stack is a combination of software and hardware implementations of the actual protocols defined in a standard as well as a definition of how devices using a certain standard should communicate with each other based on the said standard. Figure \ref{fig:LiteratureReview-Figure2} has some protocols above and below a comparatively thicker line labelled HCI. HCI stands for the Host Controller Interface. All protocols above the HCI line are included in the host’s device software package. All protocols below the HCI are built into the Bluetooth microchip. The next few bullets discuss the Bluetooth protocol stack from the bottom up. They also define the abbreviations used in the figure.
		
		\begin{itemize}
			\item \textbf{Bluetooth radio.} It transmits data in the form of bits by using a RF. This functionality is defined in the radio layer. Bluetooth radio systems generally use the Gaussian Frequency Shift Keying (GFSK) technique to transmit and receive RFs.
			\item \textbf{Baseband.}  This layer does frequency hopping for interference mitigation, medium access control, and data packet formation. In addition, the Baseband layer also control link, channel, and error correction and flow control.
			\item \textbf{Link Manager (LM).} This layer acts as a go-between for the application and the link controller in the local Bluetooth device. Remember, the Baseband layer does link control. 
			\item \textbf{Audio.} The audio layer is almost on the same level with the LM layer. However, Audio is separated from LM so as to avoid the overhead of upper layer protocols. This is important since the Audio layer hosts protocols used to provide real time two way voice communication. The separation of Audio from LM ensures voice communication does to experience lag due to LM protocols.
			\item \textbf{Logical Link Control and Adaptation Protocol (L2CAP).} This protocol is on a layer of its own. It normally resides on the host. L2CAP acts as a conduit for data on the connection link between the Baseband and host applications. L2CAP is used to ensure both connection-oriented and connection-less services. This protocol also initiates security services for any Bluetooth communication session.
			\item \textbf{Radio Frequency COMMunication (RFCOMM).} This is a transport protocol that is used to emulate RS-232 serial ports. It enables Bluetooth devices to connect with external gadgets such as printers and scanners.
			\item \textbf{Telephony Control Specification (TCS).} This protocol defines the call control signalling needed for the establishment and/or release of speech and data calls between Bluetooth devices. It also proved functionality for exchanging signalling information that is not related to ongoing calls.
			\item \textbf{Service Discovery Protocol (SDP).} This protocol is essential since it discovers the Bluetooth services available within the RF proximity and determines the characteristics of the available services. SDP is what allows Bluetooth devices to form ad-hoc, or peer-to-peer, networks.
			\item \textbf{Object EXchange Protocol (OBEX).} OBEX is used to exchange objects between Bluetooth devices. These objects include calendar notes, business cards, and data files. The exchange is done based on a client-server model.
			\item \textbf{Transport Control Protocol/Internet Protocol (TCP/IP).} This well-known protocol provides a reliable stream of data to Bluetooth applications from the RFCOMM layer.
			\item \textbf{ATtention (AT) commands.} These are not protocols as such but are a set of commands used in general telecommunications to produce commands for management of communication sessions.
			\item \textbf{Applications.} These are the Bluetooth applications used by end users.
		\end{itemize}
		
		Data from a sending Bluetooth device traverses the protocol stack from top to bottom – changing from intelligible information to bits. At the receiver’s end, the data traverses the protocol stack from bottom to top – changing from ones and zeros to data the end user can understand.
		
		Bluetooth operates within the Industrial, Scientific, and Medical (ISM) RF band. This section of the electromagnetic spectrum ranges from 2,400 to 2,483.5 MHz and is divided into 79 channels, each with a bandwidth of 1 MHz. Since the ISM band is also home to other technologies such as microwaves and Wi-Fi, it is possible that Bluetooth communication may get some interference. To avoid this, Bluetooth interfaces employ frequency hopping every few seconds. This ensures that if one channel among the 79 has interference, data can be re-sent via another channel that will likely not have interference. Bluetooth uses a hopping rate of 1600 hops per second. Its developers decided to use the Frequency Hopping Spread Spectrum (FHSS) channel management technique where sender and receiver are synchronized to know which channels they will be hopping to at any given moment during their communication. FHSS leads to efficient channel use and is not affected by the distance between sender and receiver. This is unlike the other common channel management method: the Direct Sequence Spread Spectrum (DSSS) technique.
		
		Bluetooth usually operates on a ‘master-slave’ concept. The master device works as the moderator during communication between itself and the slave as well as among slaves themselves. For devices to connect to each other, Bluetooth demands that the two share secret codes referred to as PINs. Successful PIN exchange leads to two devices being connected over Bluetooth --- a process referred to as ‘pairing.’
		
		Bluetooth has a range of up to around 30 metres. (Preethi, Sinha, \& Varma, 2012) 
		
		The previously mentioned Bluetooth study and Bluetooth security survey --– (Ibn Minar \& Tarique 2012) and  (Singh, Sharma, \& Agrawal 2011) respectively --- mention the following as some of the gadgets in which Bluetooth technology has been implemented: mobile phones, game controllers, Personal Digital Assistants(PDAs), personal computers, laptops, keyboards, mice, printers, scanners, notebooks, palmtops, cameras, and DVD players.
		
		Here are two advantages of using Bluetooth:
		
		\begin{itemize}
			\item Bluetooth is quite flexible since it operates on both the hardware and the software level. As mentioned earlier, Bluetooth rides on a radio chip and has its control and security implemented in code.
			\item Bluetooth is quite common in smart phones within Kenya. 
		\end{itemize}		
		
		Bluetooth has some disadvantages. These include, but are not restricted to:
		
		\begin{itemize}
			\item The technology having a rather small range of operation. While common Bluetooth covers a larger distance than, say, NFC, it is not an ideal solution for wireless communication over 30 metres.
			\item Security issues since it is vulnerable to sniffing and information leaks.
		\end{itemize}
		
		\item \textbf{ Wireless Fidelity (Wi-Fi) technology. }
		
		Wireless Fidelity technology, commonly known as WiFi, is a communication technology known to many. A study on Wi-Fi (Song \& Isaac, 2014) defines it as the IEEE 802.11x standard and a short-range wireless transmission technology. The study further tells that Wi-Fi is a brand held by the WiFi Alliance, whose purpose is to improve interoperability between wireless network products based on the IEEE 802.11 standard. 
		
		The concept of wireless access networks emerged in the late 1980s as a by-product of cellular wireless technology. As the demand for cellular service grew exponentially, the cost of wireless network components decreased, while the cost of setting up and maintaining conventional copper-based subscriber networks increased. (Skariah \& Suriyakala, 2013) It was time for a wireless technology to enter the communications foray. The initial Wi-Fi standard, 802.11, was released in 1997. It was improved to 802.11a in 1999. (Song \& Isaac, 2014) From then on various enhancements have been introduced in the form of new standards such as IEEE 802.11b, IEEE 802.11g, and IEEE 802.11n. The .11n standard is the most common nowadays. It operates within both the 2.4 GHz and 5 GHz frequency range with speeds of 400 to 600 Mbps. (Song \& Isaac, 2014) 
		
		Wi-Fi operates within the unlicensed radio band between 2.4 and 5 GHz. All Wi-Fi networks use contention-based Half Duplex Time Division Duplex (TDD) techniques. TDD involves vying for shared media. All devices in a Wi-Fi network attempt to use shared media (the air) at specific time intervals. Because of this operation, Wi-Fi network devices can only send or receive data at one moment. Thus they are half duplex. The contention based nature of WiFi can cause subscriber devices far from an AP to be repeatedly interrupted by closer devices. This makes services such as Voice over Internet Protocol (VoIP) or Internet Protocol Television (IPTV) – which depend on an essential constant Quality of Service (QoS) – difficult to maintain for more than a few devices. (Skariah \& Suriyakala, 2013)
		
		To reduce the limitations imposed by half duplex communication, the 802.11n WiFi standard – a common Wi-Fi standard – optimizes technology found in the physical and MAC layers of the Open Systems Interconnection (OSI) model. It does this by implementing features such as Multiple Input Multiple Output (MIMO) and MIMO-Orthogonal Frequency Division Multiplexing (MIMO-OFDM), 40 MHz channels, and short guard intervals. These combined optimizations result in enhanced throughput of up to 600 MHz for Wi-Fi-based wireless local area networks (WLANs). (Song \& Isaac, 2014)
		
		Wi-Fi uses either Direct Sequence Spread Spectrum (DSSS) or Orthogonal Frequency Division Multiplexing (OFDM) to manage the channels allocated to it in the radio band it uses. (Skariah \& Suriyakala, 2013) OFDM is the more favoured of the channel management technologies. This is because it offers high-speed transmission rates. OFDM takes a given frequency domain and divides it into orthogonal sub-channels. Each sub-channel uses a sub-carrier to modulate signals, and each sub-carrier performs transmission parallel to other sub-channels. (Song \& Isaac, 2014)
		
		Speaking of channels, Wi-Fi standards define a fixed channel bandwidth of 25 MHz for 802.11b and 20 MHz for either 802.11a or g.
		
		Skariah and Suriyakala, (Skariah \& Suriyakala, 2013), say that modulation of bit streams across WiFi networks is done using some of the following methods:
		
		\begin{itemize}
			\item \textbf{Quadrature Phase Shift Keying. (QPSK)} This is used mostly in the 802.11b standard.
			\item \textbf{Binary Phase Shift Keying (BPSK)}, used by 802.11a and g.
			\item \textbf{Quadrature Amplitude Modulation (QAM)}, which comes in two flavours --- 16-QAM and 64-QAM --- and is used by 802.11a and g.
		\end{itemize}
		
		Wi-Fi operates by having an access point (AP, also known as a hotspot) which emits Wi-Fi signals. Devices desiring to connect to a Wi-Fi network send their requests to that network’s AP. A series of handshakes takes place which mostly involve authentication. Finally, the connecting device is issued with data that will enable it to connect to the said AP. 
		
		Figure \ref{fig:LiteratureReview-Figure3} shows a simple example of how Wi-Fi networks can be deployed. 
		
		\begin{figure}
			\includegraphics[width=0.7\linewidth]{"/run/media/joshua/DATA/ICS_2406_Pictures/01LiteratureReviewAndResearchMethodology/LR_F3_A_Simple_WiFi_Deployment"}
			\caption{A Simple WiFi Deployment}
			\label{fig:LiteratureReview-Figure3}
		\end{figure}
		
		Table 1 shows how the five Wi-Fi standards mentioned so far --- 802.11, 802.11a, 802.11b, 802.11g, and 802.11n --- compare.
		
		% begin table for comparing 802.11 standards
		% 6 rows 4 columns
		\begin{table} [ht]
			
			\centering
			
			\caption{Comparing 802.11 standards}
			
			% begin tabular to draw the table
			% 2 rows 4 columns
			\begin{tabular}{c | c | c | c}
				
				
				\hline % draw a line at the top of the table
				
				& & & \\[-0.5ex] % start row 1
				
				% column        column 2				 column 3			  column 4
				\textbf{IEEE} & \textbf{Maximum Speed} & \textbf{Frequency} & \textbf{Backward} \\[1ex]
				\textbf{Standard} & \textbf{(Megabytes per second)} & \textbf{(GigaHertz)} & \textbf{Compatible with} \\[1ex]								
				
				\hline\hline % draw a double horizontal line to separate the headings from the data
				
				% finish row 1
				
				& & & \\[1ex]
				
				% start row 2
				
				% col 1   2   3     4
				802.11 & 2 & 2.4 & - \\
				
				% finish row 2	
				
				& & & \\[1ex]
				\hline
				& & & \\[1ex]
				
				% start row 3
				
				% col 1    2    3   4
				802.11a & 54 & 5 & - \\
				
				% finish row 3					
				
				& & & \\[1ex]
				\hline
				& & & \\[1ex]
				
				% start row 4
				
				% col 1    2    3     4
				802.11b & 11 & 2.4 & - \\
				
				% finish row 4
				
				& & & \\[1ex]
				\hline
				& & & \\[1ex]
				
				% start row 5
				
				% col 1    2    3     4
				802.11g & 54 & 2.4 & 802.11b \\
				
				% finish row 5
				
				& & & \\[1ex]
				\hline
				& & & \\[1ex]
				
				% start row 6
				
				% col 1    2     3           4
				802.11n & 600 & 2.4 and 5 & 802.11a/b/g \\
				
				% finish row 6
				
				& & & \\[1ex]
				\hline
				
			\end{tabular}
			% end tabular to draw the table
			
			\label{table:802.11_comparisons}
			
		\end{table} % end table for comparing 802.11 standards
		
		Various studies ( (Song \& Isaac, 2014), (Skariah \& Suriyakala, 2013), and (Banerji \& Chowdhury, 2013) ), have found that Wi-Fi can is present in devices such as personal computers, video game consoles, smart phones, tablets, printers, PDAs, and routers.
		
		Some advantages of Wi-Fi are:
		
		\begin{itemize}
			\item It has the longest range of the four common wireless networks referred to here.
			\item Wi-Fi is a feature in almost all smart phones.
		\end{itemize}
		
		Some disadvantages of Wi-Fi are:
		
		\begin{itemize}
			\item Wi-Fi is inherently insecure. Because of its huge operation range, piggybacking is very possible --- and piggybacking could lead to data sniffing. This could result in a breach in security.
			\item Wi-Fi has the highest power draw of the mentioned technologies. Some estimate Wi-Fi to use as much as 40 times more power than Bluetooth.
		\end{itemize}
		
	\end{enumerate}
	
	% begin subsection Peer-To-Peer Technologies	
	\subsection{\textbf{Peer-To-Peer Technologies}}
	
	Two peer-to-peer (P2P) technologies will be considered:
	
	\begin{enumerate}
		\item Short Message Peer to Peer (SMPP) protocol; and 
		\item Wi-Fi Direct.
	\end{enumerate}
	
	Their consideration is here below.
	
	% begin enumerate peer to peer
	\begin{enumerate}
		
		% begin SMPP
		\item \textbf{ Short Message Peer to Peer (SMPP) protocol }
		
		SMPP is a Short Message Service (SMS) protocol that is used to send messages over a TCP/IP connection. It is an open, industry standard protocol designed to offer a flexible data communication interface for the transfer of SMS data between a Message Centre (which acts as a store for SMSes) and a SMS application system (such as a system that sends bulk SMSes to subscribers). Examples of SMS application systems include External Short Message Entities (ESMEs), Routing Entities (REs), and Message Centres (MCs). As mentioned earlier, SMPP transmits messages TCP/IP. The IP link used for this can either be a leased line or the Internet. SMPP has no security measures specified, and this allows fast delivery of bulk SMSes. (Samanta, Mohandas, \& Pais, 2012) However, this is one of its major drawbacks – and will be discussed a bit later.
		
		Note: an ESME in this context of this letter refers to external sources and sinks of short messages. Such sources and sinks include Voice Processing Systems, Wireless Application Protocol (WAP) Proxy Servers, or Message Handling computers. In this document, ESME excludes Short Message Entities (SMEs) – which are located within the mobile network. An example of an SME is a Mobile Station (MS), commonly known as a mobile phone.
		SMS first appeared in Europe in 1992. It was included in Global System for Mobile (GSM) communications right from GSM’s beginning. SMS was later ported to wireless technologies such as Code Division Multiple Access (CDMA) and Time Division Multiple Access (TDMA). A standard SMS message should be a maximum 160 characters long if each character has 7 bits ( which is suitable for encoding Latin characters such as English alphabets); or a maximum 70 characters long if each character has 16 bits(suitable for encoding Unicode Universal Character Set (UCS) 2 characters such as Chinese characters). (Samanta, Mohandas, \& Pais, 2012)
		
		SMS based services have proliferated in the past few years. These services include mobile banking, delivery services, airtime status checks, and mobile ticketing. Let us take an example of a SMS sent by a user enquiring after the status of their airtime balance. The user sends a message to 144.
		
		\begin{center}
			\texttt{User SMS: "Balance" }
		\end{center}
		
		The SMS leaves from the MS via the GSM network to the SMS Center (SMSC). The SMSC serves as the point at which all SMSes sent in a mobile network arrive for processing. The SMSC sends SMSes using the “store and forward” mechanism which involves receiving a message, storing it for some time while determining its intended recipient, then forwarding the message to the identified recipient. The SMS Centre forwards the SMS to the ESME with the destination unique number “144”. At the ESME that has number “144”, the message is parsed and checked for a matching query and a response is found. This response is forwarded to the user’s MS using the path ESME to SMSC to MS. The user now knows their airtime balance. (Samanta, Mohandas, \& Pais, 2012) But where is SMPP involved in this?
		
		As mentioned in the outset, SMPP is what is used when an ESME wants to interact with the SMSC. To make such communication happen, an ESME first establishes a session then communication between ESME and SMSC is done over this session. This communication is performed usually over a TCP/IP or an X.25 connection. For TCP/IP, application port 2775 is the default port assigned for SMPP. (Samanta, Mohandas, \& Pais, 2012)
		
		Operations over SMPP can be categorized into four groups:
		
		\begin{itemize}
			\item \textbf{Session Management:}  These operations assist in the setting up of an SMPP session between an ESME and the SMSC. Operations here also provide error handling functionalities.
			\item \textbf{Message Submission:} This set of operations allows an ESME to submit messages to the SMSC.
			\item \textbf{Message Delivery:} This set of operations allows the SMSC to submit messages to an ESME. They do the inverse of the Message Submission operations.
			\item \textbf{Ancillary Operations:} The operations provide a set of additional features such as cancellation queries or message replacements. (Samanta, Mohandas, \& Pais, 2012)
		\end{itemize}
		
		ESMEs and SMSCs interact with each other by exchanging commands. Some of the important commands the two entities exchange with each other are:
		
		\begin{itemize}
			\item \textit{bind}. The purpose of this operation is to register an instance of an ESME with the SMSC system and request an SMPP session with the SMSC over a specified network connection. (Samanta, Mohandas, \& Pais, 2012)
			\item \textit{submit\_sm}. This operation is only used by an ESME to submit a short message to the SMSC for onward transmission to a specified SME. (SMSForum, 2002) 
			\item \textit{deliver\_sm}. This operation is used when an ESME wants to send message data to the SMSC. (Samanta, Mohandas, \& Pais, 2012)
			\item \textit{data\_sm}. This operation is used to transfer data between a SMSC and an ESME. The \textit{data\_sm} operation is an alternative to the \textit{submit\_sm} and \textit{deliver\_sm} operations. (SMSForum, 2002)
		\end{itemize}
		
		Figure \ref{fig:LiteratureReview-Figure4}, gotten from study of SMPP security flaws (Samanta, Mohandas, \& Pais, 2012) illustrates how SMPP messages flow.
		
		\begin{figure}
			\centering
			\includegraphics[width=0.6\linewidth]{"/run/media/joshua/DATA/ICS_2406_Pictures/01LiteratureReviewAndResearchMethodology/LR_F4_SMPP_Message_Flow"}
			\caption{SMPP Message Flow}
			\label{fig:LiteratureReview-Figure4}
		\end{figure}
		
		As noted earlier, SMPP is used among Mobile Stations, Routing Entities, SMS Centres, External Short Message Entities, Voice Processing Systems, Wireless Application Protocol (WAP) Proxy Servers, and Message Handling computers. All these are found within GSM networks.	
		
		Some of the advantages of SMPP are:
		
		% begin itemize SMPP advantages
		\begin{itemize}
			\item It is an open, accessible standard. It is also being actively supported by its originators. (SMSForum, 2002)
			\item It is flexible and thus can take care of various consumer SMS services. (SMSForum, 2002)
			\item It works over the TCP/IP suite therefore it is not limited to GSM networks. (Samanta, Mohandas, \& Pais, 2012)
			\item SMPP allows the sending of messages in bulk. (Samanta, Mohandas, \& Pais, 2012)
		\end{itemize}
		% end itemize SMPP advantages
		
		Some of the disadvantages of SMPP are:
		
		% begin itemize SMPP disadvantages
		\begin{itemize}
			\item It uses traditional SMS technology which does not provide security. Traditional SMS sends messages as plain text. (Samanta, Mohandas, \& Pais, 2012)
			\item SMPP is usually implemented in the Application layer of the TCP/IP suite therefore it assumes that reliability will be maintained in lower layers of the suite. Assumptions are risky. (Samanta, Mohandas, \& Pais, 2012)
			\item SMPP is vulnerable to a Man-In-The-Middle attack because of its lack of end to end authentication. (Samanta, Mohandas, \& Pais, 2012)
			\item Messages sent via SMPP run the risk of being tampered with because of their being plaintext. (Samanta, Mohandas, \& Pais, 2012)
		\end{itemize}
		% end itemize SMPP disadvantages
		
		% end SMPP
		
		% begin Wi-Fi Direct
		\item \textbf{ Wi-Fi Direct }
		
		According to its white paper (Wi-Fi Alliance, 2010) Wi-Fi Direct is a game changing new technology that enables Wi-Fi devices to connect directly, making it simple and convenient to do things like print, share, sync, and display. Products that have the Wi-Fi Direct functionality can be identified by looking for the Wi-Fi CERTIFIED Wi-Fi Direct designation. Such devices can connect to each other without having to join a traditional home, office, or hotspot network connection. And overview of this technology in an IEEE journal (Camps-Mur, Garcia-Saavedra, \& Serrano, 2013) states that WiFi-Direct involved enabling device to device connectivity without requiring the presence of an AP. Device to device connectivity has been possible within the 802.11 standard mentioned in the Wi-Fi section of this literature review. Such connectivity would be done by means of the ad-hoc operation mode. (Camps-Mur, Garcia-Saavedra, \& Serrano, 2013) However, ad-hoc has some challenges, such as complicated setup processes, inefficient power use, lack of extended QoS services. Wi-Fi Direct addresses these challenges and also provides a seamless way for devices with older technology to connect with Wi-Fi Direct. (Camps-Mur, Garcia-Saavedra, \& Serrano, 2013)
		
		The Wi-Fi Direct white paper (Wi-Fi Alliance, 2010) – which has already been quoted in this section and will continue to be referenced below – was published in October 2010. It is therefore assumed that the Wi-Fi Direct technology was released around October 2010.
		In a traditional Wi-Fi network, clients look for and connect to WLANs, which are created and announced by APs. This creates a clear distinction between WLAN AP and WLAN client. Each of those WLAN components has distinctly different roles and functionalities. Within Wi-Fi Direct, however, things are different. The AP and client roles are specified dynamically since Wi-Fi Direct works as a peer to peer connection between two devices over a shared Wi-Fi connection. As a result a Wi-Fi Direct device has to implement both the AP and the client role. These roles cease from being physical ones to being logical ones. A device can even run both roles at the same time, such as a moment when a device acts as the originator of a Wi-Fi Direct connection while still accessing a different Wi-Fi Direct connection. The simultaneous execution of roles can be implemented by using different frequencies (if the device has a number of physical radios) or time-sharing a single radio channel. (Camps-Mur, Garcia-Saavedra, \& Serrano, 2013) But how do Wi-Fi Direct gadgets decide which gadget takes on which role?
		
		% begin figure A P2P Deployment
		\begin{figure}
			\centering
			\includegraphics[width=0.7\linewidth]{"/run/media/joshua/DATA/ICS_2406_Pictures/01LiteratureReviewAndResearchMethodology/LR_F5_A_P2P_Deployment"}
			\caption{A P2P Deployment}
			\label{fig:LiteratureReview-Figure5}
		\end{figure}
		% end figure A P2P Deployment
		
		Wi-Fi Direct devices, or, more formally, \textbf{P2P devices}, come to a consensus on which roles to take. In order to understand how this consensus comes about, we need to first understand the structure of a Wi-Fi Direct network. This will be explained in the following paragraphs:
		P2P devices communicate by establishing \textbf{P2P Groups}, which perform the same work as traditional Wi-Fi infrastructure networks – that is, P2P Groups have a device implementing something resembling a WLAN AP as well as devices that act in a manner resembling WLAN clients. The device implementing the AP is called the \textbf{P2P Group Owner (P2P GO)} while the clients are called \textbf{P2P Clients}. As mentioned in the last few paragraphs, these roles are dynamic, therefore when 2 P2P devices discover each other they \textbf{negotiate} their roles – P2P GO or P2P Client – before the P2P Group is set up. Once the Group is established, new P2P Clients can join the group just the same way devices can join a WLAN. Figure \ref{fig:LiteratureReview-Figure5} illustrates how P2P devices can be deployed. In Figure \ref{fig:LiteratureReview-Figure5}, all the devices have Wi-Fi Direct. We assume that the users of these devices would wish to play a game on the Internet with each other. The smartphone has a 3rd Generation (3G) interface with which it connects to the Internet. The smartphone negotiates with the laptop and they both decide the former will be the P2P GO and the latter will be a P2P Client. The users connect their gaming devices to the laptop – during which process the laptop becomes the P2P GO and the gaming gadgets the P2P Clients. This figure clearly shows that it is possible for a device to act both as a P2P GO and a P2P Client.
		
		Legacy devices – Wi-Fi enabled devices that do not have the Wi-Fi Direct technology – do not formally belong to the P2P Group but can communicate with the P2P GO as long as they do not exclusively implement the 802.11b standard and support the security measures specified by the P2P GO. Such legacy devices see the P2P GO as a standard issue WLAN AP. (Camps-Mur, Garcia-Saavedra, \& Serrano, 2013)
		
		Just as traditional APs use the Dynamic Host Configuration Protocol (DHCP) to give clients IP addresses, the P2P GO assigns P2P Clients IP addresses via DHCP.
		
		There are at least three ways in which two devices can form P2P Groups (Camps-Mur, Garcia-Saavedra, \& Serrano, 2013):
		
		% begin enumerate to list the P2P Group formation methods
		\begin{itemize}
			\item \textbf{Standard}. This is the case where P2P devices first have to find each other, then negotiate over which of them will be P2P GO.
			\item \textbf{Autonomous}. This is the situation where a P2P Device creates a P2P Group on its own and immediately becomes its P2P GO. The device starts transmitting signals indicating its group’s availability. Other gadgets can discover and connect to a group set up in this manner using the traditional Wi-Fi scanning, authentication, and address configuration method.
			\item \textbf{Persistent}. Wi-Fi Direct devices can set up a P2P Group and store network credentials and assigned roles such that the next time they set up the group, devices will already know which of them is GO and which are Clients. Such a P2P Group is called persistent.
		\end{itemize}
		% end  enumerate to list the P2P Group formation methods
		
		For security, Wi-Fi Direct devices are needed to implement Wi-Fi Protected Setup (WPS). This setup method assures users of a safe connection with little intervention on their part. WPS establishes a secure connection using methods such as the introduction of a PIN in the P2P Client side, or a button push between two connecting P2P devices. WPS is based on the Wi-Fi Protected Access II (WPA2) which employs Advanced Encryption Standard Counter Block Chaining Message Authentication Protocol (AES-CCMP) for encryption and randomly generated Pre-Shared Keys (PSKs) for mutual authentication. (Camps-Mur, Garcia-Saavedra, \& Serrano, 2013)
		
		Devices which implement Wi-Fi Direct include smartphones, printers, monitors, cameras, gaming devices, digital photo frames, desktop computers, notebooks, and netbooks. (Camps-Mur, Garcia-Saavedra, \& Serrano, 2013) (Wi-Fi Alliance, 2010) There also are open source implementations of the standard which can used on WLAN hardware such as Personal Computer Memory Card International Association (PCMCIA) cards. (Camps-Mur, Garcia-Saavedra, \& Serrano, 2013)
		
		Some of Wi-Fi Direct’s advantages are:
		
		% begin Wi-Fi Direct's advantages
		\begin{itemize}
			\item Wi-Fi Direct gives its users mobility and portability. (Jichkar, 2014)
			\item Wi-Fi Direct helps people use devices immediately, thus saving time. (Jichkar, 2014)
			\item Wi-Fi Direct is easy to use. Setting it up is quite hassle free. (Jichkar, 2014)
			\item Wi-Fi Direct provides simple, secure connections. (Jichkar, 2014)
			Wi-Fi Direct’s disadvantages include:
		\end{itemize}
		% end Wi-Fi Direct's advantages
		
		Wi-Fi Direct’s disadvantages include:
		
		% begin Wi-Fi Direct's disadvantages
		\begin{itemize}
			\item As with all network connections, allowing Wi-Fi Direct links with untrusted devices can result in breaches of security. (Wi-Fi Alliance, 2010)
			\item Implementing both client- and server-side code makes Wi-Fi Direct a little heavier than traditional WLAN technology. (Camps-Mur, Garcia-Saavedra, \& Serrano, 2013)
			\item Setting up P2P Groups calls for substantial communication between devices. This takes time. 
		\end{itemize}
		% end Wi-Fi Direct's disadvantages
		
		% end Wi-Fi Direct
		
	\end{enumerate} % end enumerate peer to peer
	
	% end subsection Peer-To-Peer Technologies	
	
	% begin subsection Audio Encoding Techniques
	\subsection{\textbf{Audio Encoding Techniques}}
	
	Speech encoding is the process of compacting a speech signal so that it can be transmitted with a substantially smaller memory. (Choudhary \& Kumar, 2014) Encoding is needed because space is one of the things we have only a finite amount of. Of course, speech that is encoded at its source will need to be decoded at its destination. Because of this, speech encoding tends to refer to the process of encoding and decoding speech. The word “codec” is used to denote an encoder and a decoder. Figure \ref{fig:LiteratureReview-Figure6}, based on an article about AMR coding (Choudhary \& Kumar, 2014), is an abstraction of the speech encoding and decoding system.
	
	Speech coding is a lossy kind of coding. This means that the output signal sounds close to the input but is not a one to one mapping of the input. 
	
	The most widely used type of speech coding is the Adaptive Multi-Rate (AMR) coding. (Choudhary \& Kumar, 2014) It was adopted by the 3rd Generation Partnership Project (3GPP) in 1988. AMR uses the Algebraic Code Excited Linear Prediction (ACELP) algorithm for voice coding. It is an improvement of the Code Excited Linear Prediction (CELP) algorithm. CELP uses analysis by synthesis to encode voice by perceptually optimising the decoded signal is a closed loop system. By doing this, CELP-based coders produce good quality output even with low bit rates. Its drawback is a signal delay of 50 milliseconds. This delay was addressed by ACELP, which uses specific algebraic structures in its codebooks to process input signals. The result is CELP-quality output with a signal delay of about 2 milliseconds. (Choudhary \& Kumar, 2014)
	
	Audio in AMR is further processed using:
	
	% begin itemize for AMR audio processing
	\begin{itemize}
		\item A \textbf{Voice Activity Detector (VAD)} which differentiates speech from silence;
		\item \textbf{Comfort Noise Generation (CNG)} which generates some background static on purpose to counter the effects of suddenly swinging from silence to speech; and
		\item \textbf{Discontinuous Transmission (DTX)} which controls the transmitter so that it does not use the battery during times of silence.
	\end{itemize}
	% end itemize for AMR audio processing
	
	AMR technologies are used in GSM networks (Choudhary \& Kumar, 2014), during conference calls, in Universal Terrestrial Radio Access Network (UTRAN) devices, and in Enhanced GSM Data Environment (EDGE) devices. (Birkehammar, Bruhn, Eneroth, Hellwig, \& Johansson, 2006)
	
	AMR comes in two forms, the latter being an improvement of the former. These forms are:
	
	% begin figure The Encoding Decoding Process
	\begin{figure}
		\centering
		\includegraphics[width=0.6\linewidth]{"/run/media/joshua/DATA/ICS_2406_Pictures/01LiteratureReviewAndResearchMethodology/LR_F6_The_Encoding_Decoding_Process"}
		\caption{The Encoding Decoding Process}
		\label{fig:LiteratureReview-Figure6}
	\end{figure}
	% end figure The Encoding Decoding Process
	
	% begin enumerate AMR forms
	\begin{enumerate}
		\item Adaptive Multi-Rate Narrowband (AMR-NB) encoding; and 
		\item Adaptive Multi-Rate Wideband (AMR-WB) encoding.
	\end{enumerate}
	% end enumerate AMR forms
	
	% begin enumerate AMR-NB and AMR-WB
	\begin{enumerate}
		% begin item AMR-NB 
		\item \textbf{ Adaptive Multi-Rate Narrowband (AMR-NB) encoding}
		
		This coding technology was the first AMR one. It operates within a bandwidth of 200 – 3400 Hz and has a total of eight rates. (Choudhary \& Kumar, 2014) It works in two rates:
		
		% begin itemize rates
		\begin{itemize}
			\item \textbf{Full-rate} – with a bit rate of 22.8 Kilobits per second (Kbps) and eight of the eight rates available; or
			\item  \textbf{Half-rate} – with a bit rate of 11.4 Kbps and six of the eight rates available
		\end{itemize}
		% end itemize rates
		
		Some of the advantages of AMR-NB are:
		
		% begin itemize advantages
		\begin{itemize}
			\item It saves space and memory for long distance communications. (Choudhary \& Kumar, 2014)
			\item It is supported by active standards bodies such as the European Telecommunications Standards Institute (ETSI) and the 3GPP. (ETSI, 2002) It has functionality to save phone battery life – what with VAD and DTX. (ETSI, 2002)
		\end{itemize}
		% end itemize advantages
		
		Some of AMR-NB’s drawbacks are:
		
		% begin itemize disadvantages
		\begin{itemize}
			\item It is vulnerable to frame stealing. (ETSI, 2002)
			\item Technologies such as VAD, DTX, and CNG demand space and processing power. (ETSI, 2002)
			\item Transmission errors can result in the loss of frames. (ETSI, 2002) 
		\end{itemize}
		% end itemize disadvantages
		
		% end item AMR-NB 
		
		% begin item AMR-WB 
		\item \textbf{ Adaptive Multi-Rate Wideband (AMR-WB) encoding}
		
		Wideband AMR was specified as an improvement to AMR-NB, AMR-WB extends mobile phone bandwidth from 200 – 3400 Hz to 50 – 7000 Hz. This results in an audio output of higher quality, intelligibility, and naturalness than that of the earlier technology. AMR-WB has been standardized by 3GPP for GSM and Wide Code Division Multiple Access (WCDMA) 3G systems. (Byun, Eo, Bum, \& Minsoo, 2005)
		
		A lab test done by Ericsson showed that AMR-WB outperformed every AMR-NB mode up to 12.2 Kbps. Even in error-prone GSM channels the former technology was better than the latter. Ericsson thus believes the adoption of AMR-WB will result in positive changes in calling patterns. (Birkehammar, Bruhn, Eneroth, Hellwig, \& Johansson, 2006)
		
		AMR-WB operates similar to AMR-NB but has a greater bandwidth.  It also offers more signal processing functionalities in the form of in-band signalling, fixed rate speech, link adaptation and fixed channel codec modes. It splits the available bandwidth into smaller sections thus reducing coding intricacies and increasing optimality. (Choudhary \& Kumar, 2014)
		
		AMR-WB within a GSM network operates based on one of two 3GPP standards (Birkehammar, Bruhn, Eneroth, Hellwig, \& Johansson, 2006):
		
		% begin enumerate AMR-WB operations
		\begin{enumerate}
			\item \textbf{Tandem-Free Operation (TFO)}, an in band signalling protocol that allows voice codec parameters to pass unmodified through Pulse Code Modulation (PCM) links in traditional voice networks.  TFO preserves quality but does not reduce transport bit rate within the network.
			\item \textbf{Transcoder-Free Operation (TrFO)}, which is a combination of out-of-band transcoder control (OoBTC) signalling and enhanced transport technology. This combination ensures that the voice signal encoding in the transmitting MS is transported without modification to the receiving MS. TrFO is initially more expensive to lay out but the OoBTC feature guarantees a higher success rate for AMR-WB calls.
		\end{enumerate}
		% end enumerate AMR-WB operations
		
		Figure \ref{fig:LiteratureReview-Figure7} shows a sketch graph of how AMR-NB and AMR-WB compare in terms of vocal spectrum frequencies. The figure clearly illustrates that Wideband coding capture more voice frequency than AMR-NB.
		
		% begin figure A Sketch Graph of the Pass band Characteristics of AMR Technologies
		\begin{figure}
			\centering
			\includegraphics[width=0.7\linewidth]{"/run/media/joshua/DATA/ICS_2406_Pictures/01LiteratureReviewAndResearchMethodology/LR_F7_A_Sketch_Graph_of_the_Pass_band_Characteristics_of_AMR_Technologies"}
			\caption{A Sketch Graph of the Pass band Characteristics of AMR Technologies}
			\label{fig:LiteratureReview-Figure7}
		\end{figure}
		% end figure A Sketch Graph of the Pass band Characteristics of AMR Technologies
		
		Apart from the advantages that come with being an AMR coding technology, AMR-WB’s advantages include:
		
		% begin itemize advantages
		\begin{itemize}
			\item A higher voice quality.
			\item It allows end users of mobile phones to have privacy, discretion, and comfort when making phone calls. (Birkehammar, Bruhn, Eneroth, Hellwig, \& Johansson, 2006)
			\item It codes a higher voice range as shown in Figure \ref{fig:LiteratureReview-Figure7}. 
		\end{itemize}
		% end itemize advantages
		
		Some of AMR-WB’s pitfalls are:
		
		% begin itemize disadvantages
		\begin{itemize}
			\item It is still relatively new in the phone communication industry. This is why TFO and TrFO are needed to incorporate AMR-WB into GSM networks. (Birkehammar, Bruhn, Eneroth, Hellwig, \& Johansson, 2006)
			\item The improvements AMR-WB has over AMR-NB need more processing. (Choudhary \& Kumar, 2014)
		\end{itemize}
		% end itemize disadvantages
		
		% end item AMR-WB 
		
	\end{enumerate}
	% end enumerate AMR-NB and AMR-WB
	
	% end subsection Audio Encoding Techniques
	
	% begin subsection Audio File Formats
	\subsection{\textbf{Audio File Formats}}
	
	After audio is decoded, it needs to be played. Operating systems will need to know which files are audio files so as to play them. Audio material coded by Motion Picture Experts Group (MPEG) – that is, audio material having an .mp3 or an .aac extension – has spread widely in the Internet since 1995. Almost everyone on the planet has heard or is owning some audio file with the abovementioned extension. The initial MPEG audio file format was defined in 1992. Since then, research on audio file formats has increased tremendously, giving rise to new and better ways of saving high quality sound files in minimal space. (Brandenburg, 1999)
	
	MPEG audio technologies are based on the principle of perceptual coding. This idea is based on psychoacoustics, which is the study of how the human mind responds to sound. It turns out that some inaudible edits of a piece of music do not affect how we hear it. Perceptual coding capitalizes on this phenomenon to produce music files whose file size has been compressed without compromising the sound of the music they store. It has to be noted that perceptual coding does not perform lossless compression – the result of perceptual encoding is a file that is smaller than the original. The catch is that the human ear will not get the difference between the original Song \& its perceptually encoded form. 
	
	A perceptual audio codec is made up of:
	
	% begin itemize codec contents
	\begin{itemize}
		\item A \textbf{filter bank} which analyses and decomposes the input signal into sub-sampled spectral components during encoding; and synthesises the spectral components into an audio signal in the decoder. (Brandenburg, 1999)
		\item A \textbf{perceptual model} which uses either the time domain input signal and/or the analysis filter bank to come up with an estimate of a masking threshold from rules defined by psychoacoustics. (Brandenburg, 1999) The masking threshold is the threshold above which changes in the audio file will be noticeable by human ears. The perceptual model is only found in the encoding section since it is only the input audio that needs to be encoded perceptually.
		\item \textbf{Quantization and coding} mechanisms. Quantization is the process of converting the sampled values of an audio signal into bit representations. Quantization introduces some noise in the spectral components. Coding involves assigning each of the bit representations gotten from quantization a binary code. Coding is usually done to keep quantization noise below the masking threshold.   In the encoder the spectral components are quantized and then coded. In the decoder, the inverse is done. The input bitstream will be decoded (the inverse of the coding just defined) and then it will undergo inverse quantization to produce the corresponding spectral components.
		\item \textbf{Encoding and decoding} mechanisms. In the encoder, a bitstream formatter is used to arrange the bits gotten from quantized and coded spectral components.  The assembled bits are then sent out of the perceptual encoding system. In the decoder, a bitstream is received, decoded into its quantized form, and released for inverse quantization.
	\end{itemize}
	% end itemize codec contents
	
	Figures \ref{fig:LiteratureReview-Figure8} and \ref{fig:LiteratureReview-Figure9} show how a perceptual coding system encodes and decodes audio data respectively.
	
	% begin figure A Simple Perceptual Encoding System
	\begin{figure}
		\centering
		\includegraphics[width=0.7\linewidth]{"/run/media/joshua/DATA/ICS_2406_Pictures/01LiteratureReviewAndResearchMethodology/LR_F8_A_Simple_Perceptual_Encoding_System"}
		\caption{A Simple Perceptual Encoding System}
		\label{fig:LiteratureReview-Figure8}
	\end{figure}
	% end figure A Simple Perceptual Encoding System
	
	% begin figure A Simple Perceptual Decoding System
	\begin{figure}
		\centering
		\includegraphics[width=0.7\linewidth]{"/run/media/joshua/DATA/ICS_2406_Pictures/01LiteratureReviewAndResearchMethodology/LR_F9_A_Simple_Perceptual_Decoding_System"}
		\caption{A Simple Perceptual Decoding System}
		\label{fig:LiteratureReview-Figure9}
	\end{figure}
	% end figure A Simple Perceptual Decoding System
	
	MPEG audio file formats are used to process music in audio players, mobile phones (Mehta \& Kharote, 2014), High Definition Television (HDTV), videos (Brandenburg, 1999), personal computers (Youngseok \& Jongweon, 2014), digital homes, streaming applications, and the Internet (Geiger, et al., 2007) 
	
	We will consider two common MPEG audio file formats in this letter:
	
	% begin enumerate MPEG file formats
	\begin{enumerate}
		\item Motion Picture Experts Group-1/2 Layer-3 (MP3); and
		\item Advanced Audio Coding. (AAC) 
	\end{enumerate}
	% end enumerate MPEG file formats
	
	% begin enumerate MP# and AAC
	\begin{enumerate}
		
		% begin item MP3
		\item \textbf{Motion Picture Experts Group-1/2 Layer-3 (MP3)}
		
		This standard defined a data representation having a number of options (Brandenburg, 1999) such as:
		
		% begin itemize MP3 options
		\begin{itemize}
			\item \textbf{Operation mode}. This enables MP3 audio to work in both mono and stereo mode.
			\item \textbf{Sampling rates}. This helps MP3 audio to work on a couple of sampling frequencies such as 44.1 KHz for MPEG-1 and 22.05 KHz for MPEG-2.
			\item \textbf{Bit rate}. This option allows the implementers of MP3 standards to decide the bit rate of the audio compressed by their implementation of the standard.
		\end{itemize}
		% end itemize MP3 options
		
		MP3 encoding and decoding follows the perceptual model. Below are some details about how MP3 encodes audio (Brandenburg, 1999).
		
		% begin enumerate MP3 encoding
		\begin{enumerate}
			\item The \textbf{filter bank} is built by cascading two kinds of filter banks – the \textbf{polyphase filter bank} and then a \textbf{Modified Discrete Cosine Transform (MDCT)}. The polyphase filter bank is what is used in MPEG-1/2 Layer-1 and MPEG-1/2 Layer-2. It is used in Layer-3 to make it a little similar to the previous two layers. The MDCT ensures the audio data is stored in an overlapping manner so as to save space.
			\item The \textbf{perceptual model} is what determines the quality of the encoder implementation by setting the \textbf{level of noise allowed} for each partition of encoded audio. 
			\item Quantization is done using a \textbf{power-law quantizer} and coding is done using \textbf{Huffman coding}. Before quantization is done for a given audio data block from the perceptual model, the optimum gain and noise control for that data block are determined using two loops, one inside the other. 
			
			% begin itemize loops
			\begin{itemize}
				\item The inner loop determines the quantization rate by adjusting the number of bits resulting from a coding operation until the resulting bits are equal to the number of bits allowable for each coded audio data block. This loop is called the \textbf{rate loop}.
				\item The outer loop shapes the quantization noise according to the masking threshold set by the perceptual model. Each data block has a scale factor attached to it. Each scale factor starts off with its value being 1.0. If the quantization noise for a particular data block is found to be higher than the threshold then that block’s scale factor is adjusted to reduce the quantization noise. Since getting a smaller quantization noise needs a larger number of quantization steps and thus a higher bit-rate, the rate loop has to be repeated every time scale factors change. The outer loop manages the quantization noise and is thus called the \textbf{noise control loop}.
			\end{itemize}
			% end itemize loops
			
			\item At this point the \textbf{bitstream} is funnelled \textbf{out} of the encoder. Possible destinations include a remote audio output device or a local storage device.
		\end{enumerate}
		% end enumerate MP3 encoding
		
		When an MP3 bitstream gets to a decoder, it contains a sequence of data frames put one after another. Each frame corresponds to two granules, or long blocks, of audio. Each granule has exactly 576 consecutive audio samples. A granule may be further divided into three short blocks which have exactly 192 samples each. The following few steps explain how MP3 decoding works. (Mehta \& Kharote, 2014)
		
		% begin enumerate MP3 decoding
		\begin{enumerate}
			\item \textbf{Synchronization} of the decoder \textbf{with the start} of the MP3 \textbf{frame}. This is done to decode MP3 header information.
			\item \textbf{Decoding} of MP3 \textbf{side information}. This side information is found on the side of MP3 audio data and may contain scale factor selection information and block splitting information.
			\item \textbf{Decoding} of the main data for each granule. Main data includes \textbf{Huffman bits} and \textbf{scale factors}.
			\item \textbf{Inverse quantization of transform coefficients}. In the case of short blocks, transform coefficients may be re-ordered and divided into three sets of coefficients, one set for each block. An alias reduction is done for long blocks.
			\item The \textbf{Inverse Modified Discrete Cosine Transform (IMDCT)}, the inverse of the MDCT done during encoding, is applied for the transform coefficients acquired in step 4 of decoding.
			\item The inverse poly-phase filter bank, the inverse of the encoding poly-phase filter bank, is applied to IMDCT’s output to produce a full-bandwidth signal. 
		\end{enumerate}
		% end enumerate MP3 decoding
		
		MP3 has some advantages for those who use it. These include:
		
		% begin itemize MP3 advantages
		\begin{itemize}
			\item It is a very common audio file format. (Youngseok \& Jongweon, 2014)
			\item It produces audio files that are small in size and good in quality. (Mehta \& Kharote, 2014)
			\item It is an open standard and can therefore be implemented by anyone with the right skills.
		\end{itemize}
		% end itemize MP3 advantages
		
		MP3’s shortfalls consist of:
		
		% begin itemize MP3 disadvantages
		\begin{itemize}
			\item Since MP3 uses perceptual coding system, it leaves out some data from audio files coded by it. (Mehta \& Kharote, 2014) This may lead to an overall loss of quality with repeated MP3 encoding.
			\item The small size of MP3 files has contributed to a lot of illegal possession of MP3-file music. (Brandenburg, 1999) 
			\item The rate and noise control loops need tuning so as to avoid indefinite looping. (Brandenburg, 1999)
		\end{itemize}
		% end itemize MP3 disadvantages
		
		% end item MP3
		
		% begin item AAC
		\item \textbf{ Advanced Audio Coding (AAC)}
		
		AAC emerged as the spiritual successor of the very successful MP3 file format. It has been called the new all-round coder to take the mantle from MP3. (Herre \& Dietz, 2008) AAC has been developed to support functionalities such as scalability, low-delay operation, and lossless signal representation. (Herre \& Dietz, 2008) The original version of AAC was published in 1997 and finalized in 1999. (Herre \& Dietz, 2008)
		
		To encode audio, AAC does the following:
		
		% begin enumerate AAC encoding
		\begin{enumerate}
			\item It gets a PCM digital audio signal. 
			\item It uses a \textbf{MDCT filter bank} to transform the PCM signal into a spectral representation. This representation will be used to apply psychoacoustic principles and redundancy reduction algorithms. AAC uses a 1,024 spectral line MDCT filter bank to create spectra corresponding to 1,024 PCM input signals. (Geiger, et al., 2007)
			\item Quantization and coding processes in AAC are similar to those found in MP3. 
			\item  Before the audio signal is fully converted into bits, it will have passed through some new tools unique to AAC. These tools include:
			
			% begin itemize AAC tools
			\begin{itemize}
				\item \textbf{Temporal Noise Shaping (TNS)}. This tool allows AAC to shape quantization noise by doing an open loop prediction along the input signal’s frequency domain. This tool especially helps AAC to improve output quality at low bit rates. (Brandenburg, 1999)
				\item \textbf{Block switching}. Instead of using MP3’s cascading filter bank, AAC uses a standard switched MDCT filter bank with an impulse response (for short blocks) of 5.3 milliseconds at a 48 KHz sampling frequency. This is better than MP3’s 18.6 short block impulse response. (Brandenburg, 1999)
			\end{itemize}
			% end itemize AAC tools
			
		\end{enumerate}
		% end enumerate AAC encoding
		
		AAC decoding happens as follows:
		
		% begin enumerate AAC decoding
		\begin{enumerate}
			\item The bitstream is received by the codec.
			\item It goes through \textbf{inverse quantization} and then \textbf{decoding}. 
			\item Any \textbf{AAC tools} applied to the bitstream are \textbf{reversed}. Error mapping is also done at this point.
			\item The inversely quantized, decoded bitstream is passed through the \textbf{IMDCT}.
			\item A full audio signal is produced.
		\end{enumerate}
		% end enumerate AAC decoding
		
		Figures \ref{fig:LiteratureReview-Figure10} and \ref{fig:LiteratureReview-Figure11} show a simplified AAC encoder and decoder respectively. They attempt put into pictures what the above few lines have put in words.
		
		% begin figure A Simplified Structure of an AAC Encoder
		\begin{figure}
			\centering
			\includegraphics[width=0.7\linewidth]{"/run/media/joshua/DATA/ICS_2406_Pictures/01LiteratureReviewAndResearchMethodology/LR_F10_A_Simplified_Structure_of_an_AAC_Encoder"}
			\caption{A Simplified Structure of an AAC Encoder}
			\label{fig:LiteratureReview-Figure10}
		\end{figure}
		% end figure A Simplified Structure of an AAC Encoder
		
		% begin figure A Simplified Structure of an AAC Decoder
		\begin{figure}
			\centering
			\includegraphics[width=0.7\linewidth]{"/run/media/joshua/DATA/ICS_2406_Pictures/01LiteratureReviewAndResearchMethodology/LR_F11_A_Simplified_Structure_of_an_AAC_Decoder"}
			\caption{A Simplified Structure of an AAC Decoder}
			\label{fig:LiteratureReview-Figure11}
		\end{figure}
		% end figure A Simple Perceptual Decoding System
		
		In addition to the advantages brought about by being based on the MP3 standard, some advantages of using AAC codecs are:
		
		% begin itemize AAC advantages
		\begin{itemize}
			\item AAC encoding and decoding is flexible. That has helped to develop more refined forms of AAC codecs from the basic AAC model. (Geiger, et al., 2007)
			\item AAC has a higher coding efficiency than MP3 due to the use of prediction. (Brandenburg, 1999)
			\item They have near-lossless audio representation. (Geiger, et al., 2007)
		\end{itemize}
		% end itemize AAC advantages
		
		Since AAC inherits the MP3 model, it also inherits MP3’s disadvantages. To add to these are the following demerits unique to AAC:
		
		% begin itemize AAC disadvantages
		\begin{itemize}
			\item AAC has more features included in it such as prediction and TNS. These increase the processing power needed to encode and decode audio using AAC. (Brandenburg, 1999)
			\item The features mentioned above result in higher quality audio. However, this higher quality output needs more space on memory. (Brandenburg, 1999)
		\end{itemize}
		
		% end item AAC
		
	\end{enumerate}
	% end enumerate MP3 and AAC
	
	% end subsection Audio File Formats
	
	
	\begin{center}
		---
	\end{center}
	
	After considering the technologies discussed above, it was decided that the project will use the following technologies to implement the project idea:
	
	% begin itemize chosen technologies
	\begin{itemize}
		\item Wi-Fi for wireless communication because of its proliferation and our familiarity with it;
		\item AMR-WB for speech coding since it is the newest and most preferred encoding technique in Android; and
		\item AAC as the audio file format since it is also the most recent most common audio file format.
	\end{itemize}
	% end itemize chosen technologies
	
	From now on, the project will focus on transferring audio information over wireless technology.
	
	No peer to peer technology was chosen since we desire to implement peer to peer over Wi-Fi during this project. Also, as noted in its disadvantages, Wi-Fi Direct is not so common yet.
	
	A lot of words have been written in this project concerning various technologies that assist in audio exchange, what with Bluetooth, Wi-Fi, and AMR. But one major, very widespread technology has not been touched on: GSM.
	
	What is GSM? How does it work? Does it have any advantages? Disadvantages? And, most importantly, how does GSM compare with what this project is trying to implement? These questions will be answered over the next few pages.
	
	% begin subsection GSM
	\subsection{\textbf{GSM}}

	In the beginning of the 1980s there were several systems for mobile communications in Europe. There was an acute need for a common communications system. (Willassen, 2003) In 1982, a group of European states came up with a new standards organization, “Groupe Speciale Mobile” (GSM). Its work was to develop a communications standard common for the member countries. (Willassen, 2003) In 1988, GSM was put in the ETSI, making GSM a standard for all telecommunications across Europe. (Willassen, 2003)

	Unlike the other telecommunications systems that came up during that time, GSM was, and still is, a fully digital system allowing speech and data transfer as well as roaming across networks and countries (Willassen, 2003). Currently, GSM means “Global System for Mobile communication” and is a trademark. The ETSI group working on telecommunications standards has been renamed SMG (Special Mobile Group) so as to avoid confusion between it and GSM (Willassen, 2003).

	Table 2, gotten from an International Engineering Consortium (IEC) GSM guide (IEC, 1999), shows some of GSM’s milestones between the late 1980s and early 1990s.

	% begin table for GSM milestones
	\begin{table}[ht]

		\centering 
		
		\caption{GSM Milestones}
		
		\begin{tabular}{|c|c|}
		
			\hline & \\ [0.5ex] 
			\textbf{Year} & \textbf{Milestone}  \\ 
			\hline
			& \\ [0.5ex] 
			1982 & GSM formed \\
			& \\ [0.5ex] 
			1986 & Field test \\ 
			& \\ [0.5ex] 
			1987 & TDMA chosen as access method \\
			& \\ [0.5ex] 
			1988 & Memorandum of understanding signed \\
			& \\ [0.5ex] 
			1989 & Validation of GSM system \\
			& \\ [0.5ex] 
			1990 & Preoperation system \\
			& \\ [0.5ex] 
			1991 & Commercial system start-up \\
			& \\ [0.5ex] 
			1992 & Coverage of larger cities/airports \\
			& \\ [0.5ex] 
			1993 & Coverage of main roads \\
			& \\ [0.5ex] 
			1995 & Coverage of rural areas \\
			\hline 
			
		\end{tabular} 	
		
		\label{table:GSM_Milestones}
					
	\end{table}
	% end table for GSM milestones

	The GSM network is divided into three major systems (IEC, 1999). These are:

	% begin enumerate GSM major systems
	\begin{enumerate}
		\item The Switching System (SS);
		\item The Base Station System (BSS); and
		\item The Operation and Support System (OSS).
	\end{enumerate}
	% end enumerate GSM major systems
	
	We will consider these systems below:
		
	% begin enumerate GSM major systems discussion
	\begin{enumerate}
		
		% begin item SS
		\item \textbf{ The Switching System (SS)}
		
		It is responsible for performing call processing and subscriber-related functions (IEC, 1999). This system contains the following subsystems:
		
		% begin itemize SS subsystems
		\begin{itemize}
			\item \textbf{Home Location Register (HLR)} – This is a database used to store and manage subscriber data. (IEC, 1999)
			\item \textbf{Mobile Switching Center (MSC)} – The MSC performs the switching of phone calls. (IEC, 1999)
			\item \textbf{Visitor Location Registry (VLR)} – This is a database that has the temporary information about subscribers that is needed by the MSC in order to take care of subscribers who are visiting. (IEC, 1999) Visitor subscribers are those who are not in the HLR database. 
			\item \textbf{Authentication Center (AUC)} – This subsystem provides authentication and encryption parameters used to verify the user’s identity and ensure the confidentiality of each call. (IEC, 1999)
			\item \textbf{Equipment Identity Register (EIR)} – This is yet another database. It contains information about the identity of mobile equipment that ensures that no calls are made from stolen, unauthorized, or defective MSs. (IEC, 1999)
		\end{itemize}
		% end itemize SS subsystems
		
		% end item SS
		
		% begin item BSS
		\item \textbf{ The Base Station System (BSS)}
		
		This system performs all radio-related functions (IEC, 1999).  It has the following subsystems (IEC, 1999):

		% begin itemize BSS subsystems
		\begin{itemize}
			\item \textbf{Base Station Controller (BSC)} – This subsystem controls several Base Transmission Stations (BTSs). (BTSs will be discussed below.) The BSC handles procedures regarding call setup, location update and handover for each individual MS (Willassen, 2003).
			\item \textbf{Base Transmission Station (BTS)} – The BTS is the radio equipment needed to service each MS in the network (IEC, 1999). It contains transceivers and antennas (Willassen, 2003). In layman terms, the BTS is the booster. 
		\end{itemize}
		% end itemize BSS subsystems
		
		% end item BSS
		
		% begin item OSS
		\item \textbf{The Operation and Support System (OSS)}
		
		This is functional entity from which the network operator watches over and regulates the system. (IEC, 1999) The OSS also provides a network overview and gives support for the maintenance activities of different maintenance organizations (IEC, 1999).
		
		% end item OSS
		
	\end{enumerate}
	% end enumerate GSM major systems discussion	
	
	Figure \ref{fig:LiteratureReview-Figure12}, derived from an IEC document (IEC, 1999), shows the GSM network’s elements. A couple of items in the figure have not yet been mentioned. These are:
			
	% begin itemize unmentioned elements
	\begin{itemize}
		\item \textbf{Message Center (MXE)} – This is a node that takes care of SMS, cell broadcast, voice mail, fax mail, e-mail, and notification. (IEC, 1999) 
		\item \textbf{Mobile Service Node (MSN)} – This handles mobile intelligent network services. (IEC, 1999)
		\item \textbf{Gateway Mobile services Switching Center (GMSC)} – This node is used to provide a connection between networks. (IEC, 1999)
		\item \textbf{GSM InterWorking Unit (GIWU)} – This node provides an interface to various networks for data communication. It helps users to alternate between data and speech during the very same call. (IEC, 1999)
		\item \textbf{Public Switched Telephone Network (PSTN)} – This is the interconnection of voice-based public telephone networks in all parts of the world.
		\item \textbf{Public Land Mobile Networks (PLMNs)} – These are any wireless communications systems intended for use by subscribers on land. PLMNs may be stand-alone but are usually connected with systems such as the PSTN.
		\item \textbf{Packet Switched Public Data Network (PSPDN)} – This is a network that allows for the transfer of packet data between data networks.
		\item \textbf{Mobile Station} – This is the user equipment. It has two elements: the Mobile Equipment (ME) (the phone itself) and the Subscriber Identity Module (SIM) (Willassen, 2003). 
	\end{itemize}
	% end itemize unmentioned elements	
			
	% begin figure 12 GSM network elements
	\begin{figure}
		\centering
		\includegraphics[width=0.5\linewidth]{"/run/media/joshua/DATA/ICS_2406_Pictures/01LiteratureReviewAndResearchMethodology/LR_F12_GSM_Network_Elements"}
		\caption{GSM Network Elements}				\label{fig:LiteratureReview-Figure12}
	\end{figure}
	% end figure 12 GSM network elements
			
	When a user makes a call, their information goes through the Mobile Station to the Base Transmission Station then to the Base Station Controller. The BSC sends the information to a Mobile Switching Center which determines how to route caller information so that it can reach the individual being called (IEC, 1999). This involves checking location registers, authenticating the caller, and possibly routing the information to far away networks. On the receiving side, the receiving MSC sends the information through the BSC then to the BTS then to the called individual’s phone. Communication thus takes place.
	
	Concerning security, GSM provides authentication and encryption. These are closely related to the AUC. The user and the network have a shared secret key called the Ki (Willassen, 2003). The Ki is stored in the SIM and is not acc directly accessible to the user (Willassen, 2003).
	
	% begin itemize for authentication and encryption
	\begin{itemize}
		
		% begin item Authentication 
		\item \textbf{Authentication} 
		
			% begin itemize authentication points
			\begin{itemize}
				\item Each time the MS connects to the network, the network authenticates the user by sending a random number to the MS. The SIM then uses the A3 encryption algorithm to compute an authentication token using the random number and the Ki. The MS sends the authentication token back to the network (Willassen, 2003). The network computes an authentication token independently and then compares its own token with the one sent by the MS. If they match then the MS is authenticated. 
			\end{itemize}
			% end itemize authentication points
			
		% end item Authentication 
		
		% begin item Encryption
		\item \textbf{Encryption}
		
		% begin itemize encryption points
		\begin{itemize}
			\item Immediately after authentication, an encryption key Kc is computed (Willassen, 2003).The Kc is used to encrypt subsequent data moving from the MS to the network (Willassen, 2003).
		\end{itemize}
		% end itemize encryption points
		
		% end item Encryption
		
	\end{itemize}
	% end itemize for authentication and encryption

	GSM works based on cells. Each cell can be viewed as a geographical location in which a particular BTS’s effects are felt.  Cell terminologies include:
	
	% begin itemize cell terminologies
	\begin{itemize}
		\item \textbf{Cell radius} – The distance between a BTS and the outermost point of that BTS’s coverage range; 
		\item \textbf{Cell range} – The distance between two outermost points of a BTS’s coverage range. The cell range is twice the cell radius; and
		\item \textbf{Inter-site distance} – The distance between two BTS’s. This distance is usually three times the cell radius (ECC, 2010).
	\end{itemize}
	% end itemize cell terminologies
	
	The cell radius can vary based on network demand. Some studies have put the cell radius at about 580 metres for highly populated areas such as towns and 5000 metres for areas with low population such as the countryside (ECC, 2010). The idea of cells is illustrated in Figure \ref{fig:LiteratureReview-Figure13}, based on a GSM comparison written by the Electronic Communications Committee (ECC) (2010). Figure \ref{fig:LiteratureReview-Figure13} shows cells having a hexagonal shape. This shape is used since it is the same one used in the ECC GSM comparison guide of 2010 quoted in the previous sentence.

	% end subsection GSM
	
	% begin figure 13 GSM cells
	\begin{figure}
		\centering
		\includegraphics[width=0.6\linewidth]{"/run/media/joshua/DATA/ICS_2406_Pictures/01LiteratureReviewAndResearchMethodology/LR_F13_GSM_Cells"}
		\caption{GSM Cells}
		\label{fig:LiteratureReview-Figure13}
	\end{figure}
	% end figure 13 GSM cells
	
	GSM is used in cellular phones, microcontrollers (Amin \& Khan, 2014), and modems (Verma \& Bhatia, 2013).

	Some of the advantages of the GSM technology are:
	
	% begin itemize GSM advantages
	\begin{itemize}
		\item It is a very common technology. A source says GSM is the world’s largest system for mobile communication today (Willassen, 2003).
		\item Unlike the analogue communications it replaced, GSM uses digital technology. This means that GSM can scale effectively while keeping signalling mechanisms, interference, and switching operations at manageable levels (IEC, 1999).
		\item The GSM standard is abstracted enough to allow designers as much freedom as possible while still making it possible for the GSM operators to buy equipment from various suppliers (IEC, 1999).
		\item GSM ensures security of communication by using the Ki key as was explained earlier (Willassen, 2003).
	\end{itemize}
	% end itemize GSM advantages

	Some of GSM’s shortfalls include:

	% begin itemize GSM disadvantages
	\begin{itemize}
		\item Since it is so common, GSM can be used by criminals to harass other citizens. For example, in Kenya, cases of fraudsters staging mock kidnaps over cell phone have been on the rise. A person gets a seemingly innocent message from Customer Care saying they need him/her to switch off his/her phone for some hours because of some network maintenance. As soon as the person does this, the caller calls the victim’s relatives saying they have kidnapped the victim and need a certain ransom. Since the victim has switched his/her phone off, there is no way the relatives can contact him/her. The unwary relative(s) may end up sending the ransom money, only to learn that it was a scam.
		\item Old SIM cards have limited space to store contact and SMS information (Willassen, 2003).
		\item Ki encryption algorithms, such as A3, have weaknesses and can be cracked (Willassen, 2003). 
	\end{itemize}
	% end itemize GSM disadvantages

	Lastly, we will compare GSM with this project idea. This comparison will follow three categories: 
	
	% begin itemize GSM-project comparisons
	\begin{itemize}
		
		% begin item Distance
		\item \textbf{Distance.}
		
		% begin itemize distance comparison
		\begin{itemize}
			\item On the one hand, GSM is a global network. As mentioned earlier, GSM cells have a range of up to about 5 kilometres. These cells join up to form PLMNs, covering thousands of kilometres. To add to that, the interconnection of PLMNs means that GSM users can access other GSM users in almost any part of the world. 
			\item On the other hand, our project uses Wi-Fi. The Wi-Fi section of this letter showed that Wi-Fi is limited to about 100 metres. 
		\end{itemize}
		% end itemize distance comparison
		
		Conclusion: The project idea’s range vanishes into insignificance in the face of GSM’s reach.
		
		% end item Distance
		
		% begin item Security Issues
		\item \textbf{Security Issues.}
		
		% begin itemize security issues comparison
		\begin{itemize}
			\item On the one hand, as mentioned earlier, GSM uses the Ki, authentication algorithms, and resultant keys to ensure confidentiality of the data sent through it. However, we noted in GSM’s disadvantages that GSM’s security algorithms can be cracked. 
			\item On the other hand, as was noted in the Wi-Fi section of the project, Wi-Fi is very prone to data sniffing because if its huge operation range relative to Bluetooth and NFC. However, we noted that Wi-Fi Direct uses WPS to ensure security. WPS involves using AES-CCMP for encryption and PSKs for authentication. Our project will try to implement a version of WPS to safeguard the audio data sent between devices.  Figure \ref{fig:LiteratureReview-Figure14} shows the security options available for Wi-Fi hotspots in Android 4.1.2.
			
			% begin figure 14 Android 4.1.2 Wi-Fi Hotspot Security Options
			\begin{figure}
				\centering
				\includegraphics[width=0.4\linewidth]{"/run/media/joshua/DATA/ICS_2406_Pictures/01LiteratureReviewAndResearchMethodology/LR_F14_Android_4-1-2_Wi-Fi_Hotspot_Security_Options"}
				\caption{Android 4.1.2 Wi-Fi Hotspot Security Options}
				\label{fig:LiteratureReview-Figure14}
			\end{figure}
			% end figure 14 Android 4.1.2 Wi-Fi Hotspot Security Options
			
		\end{itemize}
		% end itemize security issues comparison
			
		Conclusion: Both GSM and this project’s idea have some security weaknesses.
					
		% end item Security Issues
		
		% begin item Cost
		\item \textbf{Cost.}
	
			% begin itemize cost comparison
			\begin{itemize}
				
				\item On the one hand, for the implementers, GSM is quite costly. The cost incurred by the implementers can be seen in the following ways: 
				
				% begin itemize GSM cost for implementers
				\begin{itemize}
					\item Getting the hardware in place is understandably expensive. 
					\item Providing the software that will perform real time switching of both voice and data costs additional time and money. 
				\end{itemize}
				% end itemize GSM cost for implementers
				
				For the users, GSM is rather cheap. All that is needed is:
				
				% begin itemize GSM cost for users
				\begin{itemize}
					\item A basic mobile phone,
					\item Some electricity to charge the phone’s battery, and 
					\item Some airtime.
				\end{itemize}
				% end itemize GSM cost for users
				
				\item On the other hand, for the implementers, this project’s idea is a bit cheaper compared to GSM.
				
				% begin itemize this project's cost for implementers
				\begin{itemize}
					\item First, the hardware needed is two Wi-Fi enabled (preferably Android) smartphones. This is way cheaper than the switches, routers, and registers of GSM.
					\item Second, the code to implement the peer to peer communication between those two devices is miles less complicated than that involved in GSM’s real time voice and data transmission.  
				\end{itemize}
				% end itemize this project's cost for implementers
								
				For the users, this project’s idea might be a bit expensive. 
						
				% begin itemize this project's cost for users
				\begin{itemize}
					\item First, the idea cannot be of any use unless one has a Wi-Fi enabled (preferably Android) smartphone. 
					\item Second, for those having such devices, this application can only work within the Wi-Fi range of maximum 100 meters. Users will have to know if other users are in that range first. 
					\item Third, most smartphones have an application used for dialling and calling. This project will make a dialling application. The hassle of navigating to this application to make a call as opposed to navigating to the default dialling application may deter some users.
					\item Fourth, the project idea uses Wi-Fi, meaning that it will draw a lot of power from devices. This might frustrate the user. In fact, Android 4.1.2 warns of the power draw, as seen in Figure \ref{fig:LiteratureReview-Figure15}.
					
					% begin figure 15 Android 4.1.2 Wi-Fi Power Draw Warning
					\begin{figure}
						\centering
						\includegraphics[width=0.5\linewidth]{"/run/media/joshua/DATA/ICS_2406_Pictures/01LiteratureReviewAndResearchMethodology/LR_F15_Android_4-1-2_Wi-Fi_Power_Draw_Warning"}
						\caption{Android 4.1.2 Wi-Fi Power Draw Warning}
						\label{fig:LiteratureReview-Figure15}
					\end{figure}
					% end figure 15 Android 4.1.2 Wi-Fi Power Draw Warning
					
				\end{itemize}
				% end itemize this project's cost for users
				
			\end{itemize}
			% end itemize cost comparison
		
		Conclusion: GSM and the project’s idea somehow cancel each other out in terms of costs. With the current trend towards smartphones and more long-lasting batteries, however, the project’s idea might edge GSM ever so slightly in this aspect.
		
		% end item Cost	
	
	\end{itemize}
	% end itemize GSM-project comparisons
	
	In summary, GSM and its derivatives will continue to be the go-to technology for mobile communication for the foreseeable future. This project does not intend to replace GSM. This project intends to provide cheaper communication over short distances among devices that are Wi-Fi enabled. This provision can be used in ways such as:
	
	% begin itemize uses of this project
	\begin{itemize}
		\item Making \textbf{phone calls.} Two devices in the same range can send and receive audio data simultaneously, thus a phone call. 
		\item Creating \textbf{public address systems in small rooms}. This is possible by having only one user speaking into a device and having the other users listening from a second device. It is our hope that the project will be expanded to allow for a point-to-multipoint architecture that will really implement this public address system functionality.
		\item \textbf{Real time recording and streaming.} One device can be put in a room and another in an adjacent room. The device in the adjacent room can play all the sounds in the first room as soon as they happen. In this way, a phone can be used as a bug, or listening device.
	\end{itemize}
	% end itemize uses of this project
	
	The above functionalities may make the project idea very useful in companies that have a small size since the project works within a 100 metre radius.
	
	The project is not planning to limit itself to just phone calls.
	
	% end Literature Review
	% =====================
	
	% begin Research Methodology
	% ==========================
	\newpage
	\section{\textbf{ Research Methodology} } 
	
	At least three methods of research will be used in this project:
	
	% begin enumerate research methods
	\begin{enumerate}
		\item \textbf{Experimentation};
		\item \textbf{Web Search}; and
		\item \textbf{Interviews.}
	\end{enumerate}
	
	These methods are discussed here underneath.
	
	% begin enumerate research methods
	\begin{enumerate}
		
		% begin item Experimentation
		\item \textbf{Experimentation.}
		
		This which will be used to test whether the project’s implementation will work on actual devices. Experimentation will be the research method most extensively used in this project.
		
		At least three experiments are planned. These are:
		
		% begin enumerate each experiment
		\begin{enumerate}
			\item Device connection and communication;
			\item Effect of distance on communication; and
			\item Effect of amount of audio data on communication.
		\end{enumerate}
		% end enumerate each experiment
		
		Here is how the three experiments are to be executed.
		
		% begin enumerate experiment description
		\begin{enumerate}
			% begin item Device connection and communication
			\item \textbf{Device connection and communication}
			
			\emph{\underline{\textsf{Premise}}}
			
			The idea of this experiment is to try to see if two Android smart phones can connect and communicate via Wi-Fi without using any third party intermediary devices.
			
			Figure \ref{fig:LiteratureReview-Figure16} shows what we are trying to achieve.
			
			% begin figure 16 Establishment of Communication
			\begin{figure}
				\centering
				\includegraphics[width=0.6\linewidth]{"/run/media/joshua/DATA/ICS_2406_Pictures/01LiteratureReviewAndResearchMethodology/LR_F16_Establishment_of_Communication_Between_Two_Android_Devices"}
				\caption{Establishment of Communication Between Two Android Devices}
				\label{fig:LiteratureReview-Figure16}
			\end{figure}
			% end figure 16 Establishment of Communication
			
			\textsf{\underline{\emph{Inputs}}}
			
			These include:
			
			% begin itemize Inputs
			\begin{itemize}
				\item Relevant code.
				\item Commands to connect the devices.
			\end{itemize}
			% end itemize Inputs
			
			\emph{\underline{\textsf{Tools}}}
			
			These will be two Android smart phones and an Integrated Development Environment (IDE).
			
			\emph{\underline{\textsf{Process}}}
			
			The process to be followed will be as follows:
			
			% begin enumerate process
			\begin{enumerate}
				\item Execute the code for both server and client on each of the two devices.
				\item Attempt to connect one device to the other.
			\end{enumerate}
			% end enumerate process
			
			\emph{\underline{\textsf{Outputs}}}
			
			The expected outputs include:
			
			% begin itemize outputs
			\begin{itemize}
				\item A User Interface display informing us that the two Android devices have connected with each other.
				\item Sound from one device playing on the other device.
			\end{itemize}
			% end itemize outputs
			
			% end item Device connection and communication
			
			% begin item Effect of distance on communication
			\item \textbf{Effect of distance on communication}
			
			\emph{\underline{\textsf{Premise}}}
			
			The idea of this experiment is to see how much distance will affect the quality of communication. Usually, the quality of wireless communication degrades when the two communicating devices increase the distance between them. We want to establish if this is so in our system. 
			
			One of the assumptions made here is that quality goes down when only parts of a file sent are received. Therefore we will check the amount of bytes in the audio file sent from the sender and compare it with the amount of bytes in the audio file received by the receiver. Since our system will be sending audio files every second, we will need to fix the amount of audio data sent by fixing the amount of time audio will be recorded at the sender side. This time will be set at a stationary five seconds. The distance between devices will be varied from zero metres to the Wi-Fi maximum range of 100 metres.
			
			Figure \ref{fig:LiteratureReview-Figure17} shows a part of the experiment we plan to carry out. Audio is input into the smart phone on the left for a fixed five seconds. The distance between devices is then varied by adding ten metres to it after each iteration. The “Process” segment of this experiment will give more details concerning the experiment.
						
			% begin figure 17 Varying Device-to-Device Distance
			\begin{figure}
				\centering
				\includegraphics[width=0.6\linewidth]{"/run/media/joshua/DATA/ICS_2406_Pictures/01LiteratureReviewAndResearchMethodology/LR_F17_Varying_Device-to-Device_Distance"}
				\caption{Varying Device-to-Device Distance while Keeping Audio Input Fixed so as to Determine Effect of Distance on Communication }
				\label{fig:LiteratureReview-Figure17}
			\end{figure}
			% end figure 17 Varying Device-to-Device Distance
			
			\textsf{\underline{\emph{Inputs}}}
			
			There will be only one input: a five second audio input at the sender’s end.
									
			\emph{\underline{\textsf{Tools}}}
			
			The tools used will include:
			
			% begin itemize tools
			\begin{itemize}
				\item Two Android smart phones;
				\item An IDE;
				\item A timer if necessary;
				\item A 100 meter tape measure; and
				\item The method \texttt{length()} of the Java class \texttt{File}.
			\end{itemize}
			% end itemize tools
			
			\emph{\underline{\textsf{Process}}}

			The experiment will be carried out as follows:
			
			% begin enumerate process
			\begin{enumerate}
				\item Write code on the sender’s side that will record the amount of digital data gotten from converting the analogue five second audio input into a digital format. This code will involve use the \texttt{length()} method.
				\item Write code on the receiver’s side that will record the amount of data received from the sender. This code will also use \texttt{length()}.
				\item Set the devices zero metres apart.
				\item Establish a connection between the two devices.
				\item Play the five second audio into the sender device. This step can also be executed by talking into the sender device for exactly five seconds. This alternative calls for the use of a timer to ensure exactly five seconds of audio are input.
				\item The code written in part (a) should be able to record the amount of digital data gotten at the sender.
				\item The recorded data will be sent to the receiver device thanks to the application code.
				\item The code written in part (b) should be able to record the amount of digital data received at the receiver’s side.
				\item The receiver device should attempt to play the data. 
				\item The recorded sent and received data amount should now be put in permanent storage for further processing.
				\item If the current distance between devices is less than 100 metres then the distance between the two devices should be increased by ten metres and the experiment should go back to step (d). Otherwise the experiment should terminate. 
			\end{enumerate}
			% end enumerate process
							
			\emph{\underline{\textsf{Outputs}}}

			The expected output will include the following:
			
			% begin itemize outputs
			\begin{itemize}
				\item The amount of bytes lost at every distance.
				\item Hopefully a graph of the same.
			\end{itemize}
			% end itemize outputs
			
			% end item Effect of distance on communication
			
			% begin item Effect of amount of audio data on communication
			\item \textbf{Effect of amount of audio data on communication}
			
			\emph{\underline{\textsf{Premise}}}
			
			This experiment will test how well the application handles volumes of data. 
			To simulate various volumes of data we have decided to vary the amount of time audio data recorded. The time taken to record the audio data will be called the talk time. We will start with a talk time of ten seconds. The audio data will be recorded at the sender side. 
			
			However, we will keep the distance between devices fixed at 50 metres (half the maximum Wi-Fi range). We will then compare the mean lost bytes at various audio recording times with the mean lost bytes at 50 metres that was established in experiment 2. We will refer to the mean lost bytes at 50 metres as the base mean and the mean lost bytes at various audio recording times as the varied records means. The measure of how much the varied records means deviate from the base mean will tell us how communication quality is affected by amount of audio data.
			A lower variation will mean that little quality is lost with increase in audio data. This is because a variance of zero implies that all sampled data is identical, and so a variance close to zero will imply that the varied records means will be close to identical to the base mean.
			
			Figure \ref{fig:LiteratureReview-Figure18} attempts to illustrate how the experiment will work. As said in the above paragraphs, the distance between the two testing devices will be fixed at 50 metres. The figure shows that. What will be varied is the amount of time audio will be played into the sender device. Figure \ref{fig:LiteratureReview-Figure18} shows the audio input at the sender side being increased by five seconds at every iteration of the experiment. 

			% begin figure 18 Varying Audio Input
			\begin{figure}
				\centering
				\includegraphics[width=0.6\linewidth]{"/run/media/joshua/DATA/ICS_2406_Pictures/01LiteratureReviewAndResearchMethodology/LR_F18_Varying_Audio_Input"}
				\caption{Varying Audio Input while Keeping Device-to-Device Distance Fixed so as to Determine Effect of Amount of Input Data on Communication}
				\label{fig:LiteratureReview-Figure18}
			\end{figure}
			% end figure 18 Varying Audio Input
						
			\textsf{\underline{\emph{Inputs}}}

			The inputs include:

			% begin itemize Inputs
			\begin{itemize}
				\item A fixed device-to-device distance of 50 metres; and
				\item Audio.
			\end{itemize}
			% end itemize Inputs
			
			\emph{\underline{\textsf{Tools}}}
			
			The tools include:
			
			% begin itemize tools
			\begin{itemize}
				\item Two Android devices; 
				\item An IDE;
				\item A timer; and
				\item The method \texttt{length()} of the Java class \texttt{File}. 
			\end{itemize}
			% end itemize tools
			
			\emph{\underline{\textsf{Process}}}

			The experiment should happen as follows:
			
			% begin enumerate process
			\begin{enumerate}
				\item Write code on the sender’s side that will record the amount of digital data gotten from converting analogue audio input into a digital format. This code will involve using the \texttt{length()} method.
				\item Write code on the receiver’s side that will record the amount of data received from the sender. This code will also use \texttt{length()}.
				\item Set the devices such that the distance between them is equal to the fixed distance defined in the Inputs section of this experiment.
				\item Establish a connection between the two devices.
				\item Start with a talk time of ten seconds.
				\item Talk into the sender device for ten seconds. 
				\item The code written in part (a) should be able to record the amount of digital data gotten at the sender.
				\item The recorded data will be sent to the receiver device.
				\item The code written in part (b) should be able to record the amount of digital data received at the receiver’s side.
				\item Store the recorded data in permanent storage.
				\item If the talk time is less than 100 seconds then increase the talk time by ten seconds and go back to step (f). Otherwise terminate the experiment. 
				
			\end{enumerate}
			% end enumerate process
			
			\emph{\underline{\textsf{Outputs}}}

			The outputs anticipated include:

			% begin itemize outputs
			\begin{itemize}
				\item Data on the amount of data sent and received at various recording times.
				\item A graph of comparing this data with the various talk times. 
			\end{itemize}
			% end itemize outputs
			
			% end item Effect of amount of audio data on communication
			
		\end{enumerate}
		% end enumerate experiment description
				

		% end item Experimentation
		
		% begin item Web Search
		\item \textbf{Web Search.}

		Here, various websites will be visited to get solutions to project problems as well as get inspiration to work around implementation issues.
		
		Some of these sites include:
		
		% begin itemize sites
		\begin{itemize}
			\item http://stackoverflow.com/, a software developer community where programmers post their coding problems and get answers from the community;
			\item https://en.wikipedia.org, home to Wikipedia –  the free encyclopedia; and
			\item http://developer.android.com/, the official Android development site.
		\end{itemize}
		% end itemize sites
		
		We expect to use the following inputs for this method of research:
		
		% begin itemize inputs
		\begin{itemize}
			\item An internet enabled device.
			\item Questions to search answers to.
		\end{itemize}
		% end itemize inputs
		
		Below are the tools we plan to use during web search:
		
		% begin itemize tools
		\begin{itemize}
			\item An internet enabled device.
			\item A web browser. 
		\end{itemize}
		% end itemize tools
		
		Here is the data we expect to get from this research method:
		
		% begin itemize data
		\begin{itemize}
			\item Answers to the questions searched for – hopefully including snippets of code implementing those answers.Questions here might include queries such as:
			
			% begin itemize possible questions
			\begin{itemize}
				\item How is timing implemented in Android?
				\item What is the difference between using the Android Activity Constructor and using the Android Activity \texttt{onCreate} method?
				\item How do I extract an mp3 file from a byte array?
				\item How do I use Android \texttt{DialogFragments}?
				\item How do I convert an \texttt{ImageIcon} to a \texttt{BitmapDrawable} in Android?
			\end{itemize} 
			% end itemize possible questions
			
			\item More questions from those answers.
		\end{itemize}
		% end itemize data

		% end Web Search
		
		% begin item Interviews
		\item \textbf{Interviews.}
		
		These will be done in an informal setting to acquire opinions from potential end users concerning the user interface, any possible limitations, and other useful pieces of information.
				
		The target population for my interviews will generally be university students since these know how to use smart phones the most.
		
		I intend to carry out interviews after every major application update. That way I will be able to get user input more often.
				
		The expected inputs to interviews are: 
				
		% begin itemize inputs
		\begin{itemize}
			\item Preparation of interview questions.
		\end{itemize}
		% end itemize inputs
				
		Interviews might need the following tools:
				
		% begin itemize tools
		\begin{itemize}
			\item A notebook and a pen to record interviewee responses.
		\end{itemize}
		% end itemize tools
				
		We expect to get the below-mentioned data from conducting interviews:
				
		% begin itemize data
		\begin{itemize}
			\item Various varying opinions on questions asked to interviewees.
			\item More questions to ask interviewees, such as when seeking clarification.
			\item Suggestions on improvements of the system 
		\end{itemize}
		% end itemize data
				
		% end item Interviews
		
	\end{enumerate}
	% end enumerate research methods

	% end Research Methodology
	% ========================
	
	\clearpage
	
	% begin References
	% ================
	
	\newpage
	
	\section{\textbf{ References } } 
	
	\begin{thebibliography}{ 10 }
		% The 10 implies you will add upto 10 references. You can 1ncrease this number if you have more than this number.
		
		\bibitem{ Ami14 }
		
		Amin, A \& Khan, M N 2014, 'A Survey of GSM Technology to Control Remote Devices', \emph{International Journal of u- and e- Service, Science and Technology}, vol. 7, no. 5, pp. 153-162, viewed 7 December 2015, http://www.sersc.org/journals/IJUNESST/vol7\_no6/14.pdf
		
		\bibitem{ Arul13 }
		
		Arul Oli, V C K P 2013, 'Wireless Fidelity Real Time Security System', \emph{International Journal of Computer Science Trends and Technology},	vol. 1, no. 1, pp. 43-50, viewed 24 October 2015, http://arxiv.org/ftp/arxiv/papers/1405/1405.1019.pdf
		
		\bibitem{ Ban13 }
		
		Banerji, S \& Chowdhury, R S 2013, 'Wi-Fi and WiMAX: A Comparative Study', \emph{Indian Journal of Engineering}, vol. 2, no. 5, viewed 25 October 2015, http://arxiv.org/ftp/arxiv/papers/1302/1302.2247.pdf 
		
		\bibitem{ Beck08 }
		
		Beck, J \& Grajeda, T 2008,
		\emph{Lowering the Boom: Critical Studies in Film Sound}, p. 43, University of Illinois Press, Champaign, IL, USA. 
		
		\bibitem{ Bir06 }
		
		Birkehammar, C, Bruhn, S, Eneroth, P, Hellwig, K, \& Johansson, S 2006, 'New high-quality voice service for mobile networks', \emph{Ericsson Review}, vol. 3, pp. 97-100, viewed 30 October 2015, http://www.ericsson.com/ericsson/corpinfo/publications/review/2006\_03/files/2\_amrwb.pdf
		
		\bibitem{ Bra99 }
		
		Brandenburg, K. 1999, 'MP3 and AAC Explained', \emph{Audio Engineering Society Journal}, pp. 1-12, viewed 30 October 2015, https://graphics.ethz.ch/teaching/mmcom12/slides/mp3\_and\_aac\_brandenburg.pdf
		
		\bibitem{ Byu05 }
		
		Byun, K J, Eo, I S, Bum, J H, \& Minsoo, H 2005, 'An Embedded ACELP Speech Coding Based on the AMR-WB Codec', \emph{Electronics and Telecommunications Research Institute (ETRI) Journal}, vol. 27, no. 6, pp. 231-234, viewed 30 October 2015, http://koasas.kaist.ac.kr/bitstream/10203/18247/1/An\%20Embedded\%20ACELP\%20Speech\%20Coding\%20Based\%20on\%20the\%20AMR-WB\%20Codec.pdf
		
		\bibitem{ Cam13 }
		
		Camps-Mur, D, Garcia-Saavedra, A, \& Serrano, P 2013, 'Device to device communications with WiFi Direct: overview and experimentation', \emph{Wireless Communications Magazine, IEEE}, vol. 20, no. 3, pp. 96-104, viewed 9 November 2015, http://www.it.uc3m.es/pablo/papers/pdf/2012\_camps\_commag\_wifidirect.pdf
		
		\bibitem{ Cho14 }
		
		Choudhary, D \& Kumar, A 2014, 'Study and Performance of AMR Codecs for GSM' \emph{International Journal of Advanced Research in Computer and Communication Engineering}, pp. 8105-8110, viewed 30 October 2015, http://www.ijarcce.com/upload/2014/october/IJARCCE1F\%20s\%20-divu-\%20abhinav\%20-\%20STUDY\%20AND\%20PERFORMANCE\%20OF\%20AMR\%20CODECS\%20FOR\%20GSM.pdf
		
		\bibitem{ Deitel12 }
		
		Deitel, P, Deitel, H, Deitel, A \& Morgano, M 2012, \emph{Android for Programmers An App-Driven Approach}, Pearsion Education, Inc., Crawfordsville, IN, USA.
		
		\bibitem{ ECC10 }
		
		ECC 2010, \emph{Compatibility Study for UMTS Operating Within the GSM 900 and GSM 1800 Frequency Bands}, Roskilde: Electronic Communications Committee (ECC), viewed 8 December 2015, http://www.erodocdb.dk/docs/doc98/official/pdf/ECCRep082.pdf
		
		\bibitem{ ETS02 }
		
		ETSI, 2002, Universal Mobile Telecommunications System (UMTS); AMR speech Codec; General description (3GPP TS 26.071 version 5.0.0 Release 5), Sophia Antipolis Cedex, France, viewed 8 November 2015, http://www.etsi.org/deliver/etsi\_ts/126000\_126099/126071/05.00.00\_60/ts\_126071v050000p.pdf 
		
		\bibitem{ Gei07 }
		
		Geiger, R, Rongshan, Y, Herre, J, Rahardja, S, Sang-Wook, K, Xiao, L, et al 2007, 'ISO/IEC MPEG-4 High-Definition Scalable Advanced Audio Coding', \emph{Audio Engineering Society Journal}, vol. 55, pp. 27-43, viewed 30 October 2015, http://www.ece.rochester.edu/courses/ECE472/Site/Assignments/Entries/2009/1/15\_Week\_1\_files/Geiger\_2007.pdf
		
		\bibitem{ Her08 }
		
		Herre, J, \& Dietz, M 2008, 'MPEG-4 High-Efficiency AAC Coding', \emph{IEEE SIGNAL PROCESSING MAGAZINE}, pp. 137-142, viewed 30 October 2015, http://www.img.lx.it.pt/~fp/cav/ano2008\_2009/Trabalhos\_MEEC\_2009/Artigo\_MEEC\_11/pagina\%20web/pagina\%20web/referencias/artigo6.pdf
		
		\bibitem{ Ibn12 }
		
		Ibn Minar, N B \& Tarique, M 2012, 'Bluetooth Security Threats ans Solutions: A Survey', \emph{International Journal of Distributed and Parallel Systems (IJDPS)} , vol. 3, no. 1, pp. 127-148, viewed 24 October 2015, http://www.airccse.org/journal/ijdps/papers/0112ijdps10.pdf
		
		\bibitem{ ISO13 }
		
		ISO/IEC-18092 2013, 'Information technology — Telecommunications and information exchange between systems — Near Field Communication — Interface and Protocol (NFCIP-1)', Geneva, Switzerland, viewed 24 October 2015, http://standards.iso.org/ittf/PubliclyAvailableStandards/c056692\_ISO\_IEC\_18092\_2013.zip
		
		\bibitem{ Jic14 }
		
		Jichkar, B R 2014, 'Paper on Proposed System for Placing Free Call over Wi-Fi Network Using Voip and SIP', \emph{International Journal of Engineering Research and Applications}, vol. 4, no. 1, pp. 132-135, viewed 8 November 2015, http://www.ijera.com/papers/Vol4\_issue1/Version\%203/V4103132135.pdf 
		
		\bibitem{ Meh14 }
		
		Mehta, A V \& Kharote, P R 2014, 'ARM 7 Based MP3 Player', \emph{International Journal of Engineering Research and Applications}, vol. 4, no. 2, pp. 01-05, viewed 13 November 2015, http://www.ijera.com/papers/Vol4\_issue2/Version\%206/A42060105.pdf 
		
		\bibitem{ Preethi12 }
		
		Preethi, K, Sinha, A, \& Varma, N 2012, 'Contactless Communication through Near Field Communication', \emph{International Journal of Advanced Research in Computer Science and Software Engineering}, pp. 158-163, viewed 24 October 2015, http://www.ijarcsse.com/docs/papers/April2012/Volume\_2\_issue\_4/V2I40047.pdf
		
		\bibitem{ Sam12 }
		
		Samanta, S, Mohandas, R, \& Pais, A R 2012, 'Secure Short Message Peer-To-Peer Protocol', \emph{International Journal of Electronic Commerce Studies}, vol. 3, no. 1, pp. 45-60, viewed 28 October 2015, http://www.academic-journals.org/ojs2/index.php/ijecs/article/viewFile/1013/101
		
		\bibitem{ Singh11 }
		
		Singh, P, Sharma, D, \& Agrawal, S 2011, 'A Modern Study of Bluetooth Wireless Technology', \emph{International Journal of Computer Science, Engineering and Information Technology}, vol. 1, no. 3, pp. 55-63, viewed 24 October 2015, http://airccse.org/journal/ijcseit/papers/0811ijcseit06.pdf
		
		\bibitem{ Skariah13 }
		
		Skariah, M \& Suriyakala, C D 2013, 'An Exploration on Wi-Fi/802.11b and WiMAX/802.16 Networks with Performance Enhancements', \emph{International Journal of Engineering Sciences \& Research Technology}, vol. 2, no. 12, pp. 3658-3664, viewed 25 October 2015, http://www.ijesrt.com/issues\%20pdf\%20file/Archives\%202013/dec-2013/71.pdf
		
		\bibitem{ SMS02 }
		
		SMSForum 2002, 'SMPP v3.4 Protocol Implementation guide for GSM/UMTS.', viewed 28 October 2015, http://opensmpp.org/specs/smppv34\_gsmumts\_ig\_v10.pdf
		
		\bibitem{ Song14 } 
		
		Song, S \& Isaac, B 2014, 'Analysis of Wi-Fi and WIMAX and Wireless Network Coexistence', \emph{International Journal of Computer Networks \& Communications (IJCNC)}, vol. 6, no. 6, pp. 63-78, viewed 25 October 2015, http://www.ijesrt.com/issues\%20pdf\%20file/Archives\%202013/dec-2013/71.pdf
		
		\bibitem{Ver13}
		
		Verma, P \& Bhatia, J S 2013, 'Design and Development of GPS-GSM Based Tracking System with Google Map', \emph{International Journal of Computer Science, Engineering and Applications (IJCSEA)}, vol. 3, no. 3, pp. 33-40, viewed 7 December 2015, http://airccse.org/journal/ijcsea/papers/3313ijcsea04.pdf
		
		\bibitem{ Wi-10 }
		
		Wi-Fi Alliance 2010, 'Wi-Fi CERTIFIED Wi-Fi Direct™', viewed 9 November 2015, http://www.broadcom.com/blog/wp-content/uploads/2013/10/Wi-Fi-Direct-White-Paper.pdf
		
		\bibitem{Wil03}
		
		Willassen, S Y 2003, Forensics and the GSM mobile telephone system, \emph{International Journal of Digital Evidence}, vol. 2, no. 1, pp. 1-17, viewed 7 December 2015, https://www.utica.edu/academic/institutes/ecii/publications/articles/A0658858-BFF6-C537-7CF86A78D6DE746D.pdf

		\bibitem{ You14 }
		
		Youngseok, L \& Jongweon, K 2014, 'MP3 File Identification Based on Concurrence Order of Metadata', \emph{International Journal of Multimedia and Ubiquitous Engineering}, vol. 9, no. 9, pp. 41-50, viewed 25 October 2015, http://www.sersc.org/journals/IJSH/vol7\_no3\_2013/5.pdf
		
	\end{thebibliography}
	
	% end References
	% ==============
	
\end{document}



















